
\documentclass[nooutcomes]{ximera}
%\documentclass[space,handout,nooutcomes]{ximera}

% For preamble materials

\graphicspath{
  {./}
  {algorithms/}
  {../algorithms/}
}


%%% This set of code is all of our user defined commands
\newcommand{\bysame}{\mbox{\rule{3em}{.4pt}}\,}
\newcommand{\N}{\mathbb N}
\newcommand{\C}{\mathbb C}
\newcommand{\W}{\mathbb W}
\newcommand{\Z}{\mathbb Z}
\newcommand{\Q}{\mathbb Q}
\newcommand{\R}{\mathbb R}
\newcommand{\A}{\mathbb A}
\newcommand{\D}{\mathcal D}
\newcommand{\F}{\mathcal F}
\newcommand{\ph}{\varphi}
\newcommand{\ep}{\varepsilon}
\newcommand{\aph}{\alpha}
\newcommand{\QM}{\begin{center}{\huge\textbf{?}}\end{center}}

\renewcommand{\le}{\leqslant}
\renewcommand{\ge}{\geqslant}
\renewcommand{\a}{\wedge}
\renewcommand{\v}{\vee}
\renewcommand{\l}{\ell}
\newcommand{\mat}{\mathsf}
\renewcommand{\vec}{\mathbf}
\renewcommand{\subset}{\subseteq}
\renewcommand{\supset}{\supseteq}
\renewcommand{\emptyset}{\varnothing}
\newcommand{\xto}{\xrightarrow}
\renewcommand{\qedsymbol}{$\blacksquare$}
\newcommand{\bibname}{References and Further Reading}
\renewcommand{\bar}{\protect\overline}
\renewcommand{\hat}{\protect\widehat}
\renewcommand{\tilde}{\widetilde}
\newcommand{\tri}{\triangle}
\newcommand{\minipad}{\vspace{1ex}}
\newcommand{\leftexp}[2]{{\vphantom{#2}}^{#1}{#2}}

%% More user defined commands
\renewcommand{\epsilon}{\varepsilon}
\renewcommand{\theta}{\vartheta} %% only for kmath
\renewcommand{\l}{\ell}
\renewcommand{\d}{\, d}
\newcommand{\ddx}{\frac{d}{dx}}
\newcommand{\dydx}{\frac{dy}{dx}}


\usepackage{bigstrut}


\newenvironment{sectionOutcomes}{}{}

\usepackage{array}
%\setlength{\extrarowheight}{-.2cm}   % Commented out by Findell to fix table headings.  Was this for typesetting division?  
\newdimen\digitwidth
\settowidth\digitwidth{9}
\def~{\hspace{\digitwidth}}
\def\divrule#1#2{
\noalign{\moveright#1\digitwidth
\vbox{\hrule width#2\digitwidth}}}


\title{Fundamental Theorem}
\author{Bart Snapp and Brad Findell}
\begin{document}
\begin{abstract}
Problems about unique factorization.
\end{abstract}
\maketitle


%\begin{problem}
%Problem
%\begin{freeResponse}
%\begin{hint}
%Hint
%\end{hint}
%\end{freeResponse}
%\end{problem} 


\begin{problem}
Explain what the GCD of two integers is. Give some relevant and
  revealing examples/nonexamples.
\begin{freeResponse}
\begin{hint}
The GCD of $a$ and $b$ is the greatest common divisor of the two integers.  Imagine the following procedure: 
\begin{enumerate}
\item List the divisors of $a$.  
\item List the divisors of $b$.  
\item Compare the two lists to create a new list of divisors they have in common.  
\item From the new list, identify the greatest of these common divisors.  
\end{enumerate}
\end{hint}
\end{freeResponse}
\end{problem}

\begin{problem}
Explain what the LCM of two integers is. Give some relevant and
  revealing examples/nonexamples.
\begin{freeResponse}
\begin{hint}
Note: Use ellipses (i.e., three dots) to indicate a continuing pattern.

The LCM of $a$ and $b$ is the least common multiple of the two integers.  Imagine the following procedure: 
\begin{enumerate}
\item List the multiples of $a$.   
\item List the multiples of $b$.  
\item Compare the two lists to create a new list of multiples they have in common.  
\item From the new list, identify the least of these common multiples.  
\end{enumerate}
\end{hint}
\end{freeResponse}
\end{problem}

%\begin{problem}
%Consider the Diophantine equation:
%\[
%15x + 4y = 1
%\]
%\begin{enumerate}
%\item Find a solution to this equation. Explain your reasoning.
%\item Compute the slope of the line $15x + 4y = 1$ and write it in
%  lowest terms. Show your work.
%\item Plot the line determined by $15x + 4y = 1$ on graph paper.
%\item Using your plot and the slope of the line, explain how to find
%  $10$ more solutions to the Diophantine equation above.
%\end{enumerate}
%\end{problem}
%
%\begin{problem}
%Explain why a Diophantine equation 
%\[
%ax + by = c
%\]
%has either an infinite number of solutions or zero solutions.
%\end{problem}
%
%\begin{problem}
%Josh owns a box containing beetles and spiders. At the moment,
%  there are $46$ legs in the box. How may beetles and spiders are
%  currently in the box? Explain your reasoning.
%\end{problem}
%
%\begin{problem}
%How many different ways can thirty coins (nickles, dimes, and
%  quarters) be worth five dollars? Explain your reasoning.
%\end{problem}
%
%\begin{problem}
%Lisa collects lizards, beetles and worms. She has more worms
%  than lizards and beetles together. Altogether in the collection
%  there are twelve heads and twenty-six legs. How many lizards does
%  Lisa have?  Explain your reasoning.
%\end{problem}
%
%\begin{problem}
%Can you make exactly \$$5$ with exactly $100$ coins assuming you
%  can only use pennies, dimes, and quarters? If so how, if not why
%  not?  Explain your reasoning.
%\end{problem}
%
%\begin{problem}
%A merchant purchases a number of horses and bulls for the sum
%  of $1770$ talers. He pays $31$ talers for each bull, and $21$ talers
%  for each horse. How many bulls and how many horses does the merchant
%  buy? Solve this problem, explain what a \textit{taler} is, and
%  explain your reasoning---note this problem is an old problem by
%  L.\ Euler, it was written in the $1700$'s.
%\end{problem}
%
%\begin{problem}
%A certain person buys hogs, goats, and sheep, totaling $100$
%  animals, for $100$ crowns; the hogs cost him $3\frac{1}{2}$ crowns
%  a piece, the goats $1\frac{1}{3}$ crowns, and the sheep go for
%  $\frac{1}{2}$ crown a piece. How many did this person buy of each?
%  Explain your reasoning---note this problem is an old problem from
%  \textit{Elements of Algebra} by L.\ Euler, it was written in the
%  $1700$'s.
%\end{problem}

\begin{problem}
How many zeros are at the end of the following numbers:
\begin{enumerate}
\item $2^2 \cdot 5^8 \cdot 7^3\cdot 11^5$. There are $\answer{2}$ zeros. 	\begin{hint}Consider the 2s and 5s.  (Why?)\end{hint}
\item $11!$. There are $\answer{2}$ zeros.
\begin{hint}$11! = 11\cdot10\cdot9\dots2\cdot1$, and imagine its prime factorization.  There will be plenty of 2s. Count the 5s.\end{hint}
\item $27!$. There are $\answer{6}$ zeros.
 \begin{hint}If you were to write out the 27 factors, 5s are contributed by the following factors:  5, 10, 15, 20, 25.  And the 25 contributes a second 5.\end{hint}
\item $99!$. There are $\answer{19+3}$ zeros. 
  \begin{hint}Among the 99 factors, there are 19 multiples of 5 and 3 multiples of 25.\end{hint}
\item $1001!$. There are $\answer{200+40+8+1}$ zeros.
  \begin{hint}Among the 1001 factors, there are 200 multiples of 5, 40 multiples of 25, 8 multiples of 125, and 625.\end{hint}
\end{enumerate}
In each case, explain your reasoning.
\end{problem}

\begin{problem}
Decide whether the following statements are true or false. In
  each case, a detailed argument and explanation must be given
  justifying your claim.
\begin{enumerate}
\item $7|56$. \wordChoice{\choice[correct]{True}\choice{False}}
  \begin{hint}$56=7\cdot8$.\end{hint}
\item $55|11$. \wordChoice{\choice{True}\choice[correct]{False}}
  \begin{hint}But $11|55$.\end{hint}
\item $3|40$. \wordChoice{\choice{True}\choice[correct]{False}}
  \begin{hint}$40 = 3\cdot39 + 1$. Division by 3 gives remainder 1.\end{hint}
\item $100 | (2^4\cdot 3^{17} \cdot 5^2\cdot 7)$
  \wordChoice{\choice[correct]{True}\choice{False}}
  \begin{hint}$100=2^25^2$.\end{hint}
\item $5555 | (5^{20}\cdot 7^9\cdot 11^{11}\cdot 13^{23})$ 
  \wordChoice{\choice{True}\choice[correct]{False}}
  \begin{hint}$5555=5\cdot 11\cdot 101$. \end{hint}
\item $3| (3+ 6 + 9 + \cdots +300 + 303)$
   \wordChoice{\choice[correct]{True}\choice{False}}
  \begin{hint}3 divides each of the terms.\end{hint}
\end{enumerate}
\end{problem}

\begin{problem}
Suppose that 
\[
(3^5 \cdot 7^9 \cdot 11^x \cdot 13^y) | (3^a \cdot 7^b \cdot 11^{19} \cdot 13^7)
\]
What values of $a$, $b$, $x$ and $y$, make true statements? Explain
your reasoning.
\begin{itemize}
\item $a$ \wordChoice{\choice[correct]{$\ge$}\choice{$=$}\choice{$\le$}} $\answer{5}$.
\item $b$ \wordChoice{\choice[correct]{$\ge$}\choice{$=$}\choice{$\le$}} $\answer{9}$.
\item $x$ \wordChoice{\choice{$\ge$}\choice{$=$}\choice[correct]{$\le$}} $\answer{19}$.
\item $y$ \wordChoice{\choice{$\ge$}\choice{$=$}\choice[correct]{$\le$}} $\answer{7}$.
\end{itemize}
\end{problem}

\begin{problem}
Decide whether the following statements are true or false. In
  each case, a detailed argument and explanation must be given
  justifying your claim.
\begin{enumerate}
\item If $7|13a$, then $7|a$.  \wordChoice{\choice[correct]{True}\choice{False}}
  \begin{hint}Follows from Euclid's Lemma because 7 is prime.\end{hint}
\item If $6|49a$, then $6|a$.  \wordChoice{\choice[correct]{True}\choice{False}}
  \begin{hint}Because $6=2\cdot3$ is not prime, we handle its prime factors separately.  

  Because $6|49a$, it must be that $2|49a$.  Then $2|a$ by Euclid's Lemma. 
  
  Similarly, because $6|49a$, it must be that $3|49a$.  Then $3|a$ by Euclid's Lemma. 
  
  Because both $2|a$ and $3|a$, it follows that $6|a$.
  \end{hint}

\item If $10|65a$, then $10|a$. \wordChoice{\choice{True}\choice[correct]{False}}
  \begin{hint}Counterexample: $a=2$.\end{hint}
\item If $14|22a$, then $14|a$.  \wordChoice{\choice{True}\choice[correct]{False}}
 \begin{hint}Counterexample: $a=7$.\end{hint}
\item $54|931^{21}$.  \wordChoice{\choice{True}\choice[correct]{False}}
 \begin{hint}$54$ is even (i.e., it has $2$ as a factor), but $931^{21}$ is not.\end{hint}
\item $54|810^{33}$.  \wordChoice{\choice[correct]{True}\choice{False}}
 \begin{hint}From $54 = 2\cdot3^3$ and $810 = 2\cdot 5\cdot 3^4$, 
 we can see that $54|810$.\end{hint}
\end{enumerate}
\end{problem}

\begin{problem}
Joanna thinks she can see if a number is divisible by 24 by
  checking to see if it's divisible by 4 and divisible by 6.  She
  claims that if the number is divisible by 4 and by 6, then it must
  be divisible by 24.

Lindsay has a similar divisibility test for 24: She claims that if a
number is divisible by 3 and by 8, then it must be divisible by 24.

Are either correct?  Explain your reasoning.

Joanna is \wordChoice{\choice{correct}\choice[correct]{incorrect}}.  
Lindsay is \wordChoice{\choice[correct]{correct}\choice{incorrect}}.  
\begin{freeResponse}
\begin{hint}For Joanna, the least (positive) counterexample is $\answer{12}$.  Her method doesn't work because the $gcd(6,4) = \answer{2}$, so there are common multiples \emph{before} $6\cdot4$.  

Lindsay's method, works because $gcd(3,8) = \answer{1}$, so every common multiple of $3$ and $8$ is also a multiple of $3\cdot8$.  
\end{hint}
\end{freeResponse}
\end{problem}

%\begin{problem}
%Generalize the problem above.
%\end{problem}

%\begin{problem}
%Suppose that you have a huge bag of tickets. On each of the
%  tickets is one of the following numbers. 
%\[
%\{6, 18, 21, 33, 45, 51, 57, 60, 69, 84\}
%\]
%Could you ever choose some combination of tickets (you can use as many
%copies of the same ticket as needed) so that the numbers sum to 7429?
%If so, give the correct combination of tickets. If not explain why
%not.
%\end{problem}

\begin{problem}
\label{P:helper} Decide whether the following statements are true
  or false. In each case, a detailed argument and explanation must be
  given justifying your claim.
\begin{enumerate}
\item If $a^2|b^2$, then $a|b$.  \wordChoice{\choice[correct]{True}\choice{False}}
\begin{hint}However many times a prime appears in the prime factorizations of $a$, it will appear twice as many times in the prime factorization of $a^2$.  Same idea for $b$ and $b^2$. 
Because $a^2|b^2$, we know that $b^2=ka^2$ for some integer $k$.  
Both $b^2$ and $a^2$ must have an even number of factors any prime, which implies that $k$ must also have an even number of factors of that prime.  
This means that $k$ is a perfect square, which is to say it is the square of some integer $c$.  Substituting $k=c^2$, we find that $b^2=c^2a^2=(ca)^2$.  Assuming $a, b, c > 0$, we have $b=ca$, which means that $a|b$.
\end{hint}
\item If $a|b^2$, then $a|b$.  \wordChoice{\choice[correct]{True}\choice{False}}
  \begin{hint}Counterexample: $a=12$, $b=6$.\end{hint}
\item If $a|b$ and $\gcd(a,b) = 1$, then $a = 1$.  \wordChoice{\choice[correct]{True}\choice{False}}
  \begin{hint}Because $a|b$ there is an integer $k$ such that $b=ak$.  Because $\gcd(a,b) = 1$, every prime in the factorization of $b$ is not a factor of $a$ and therefore must be in $k$.  This implies that $k$ has the same prime factorization of $b$.  Assuming $a, b, k>0$, this means that $b=k$ and therefore $a=1$.\end{hint}
\end{enumerate}
\end{problem}

%\begin{problem}
%Betsy is factoring the number $24949501$. To do this, she
%  divides by successively larger primes. She finds the smallest prime
%  divisor to be $499$ with quotient $49999$. At this point she
%  stops. Why doesn't she continue? Explain your reasoning.
%\end{problem}
%
%\begin{problem}
%When Ann is half as old as Mary will be when Mary is three times
%  as old as Mary is now, Mary will be five times as old as Ann is
%  now. Neither Ann nor Mary may vote. How old is Ann? Explain your
%  reasoning.
%\end{problem}

\begin{problem}
Suppose $x$ and $y$ are integers.  If $x^2 = 11\cdot y$, what can you say about $y$? Explain your
  reasoning.
\begin{freeResponse}
\begin{hint}
Imagine comparing the prime factorizations of $x$, $x^2$, and $y$.  However many times a prime appears in the prime factorization of $x$, it will appear twice as many times in the prime factorization of $x^2$.  Because $11|(11y)$, it follows that $11|x^2$.  Because $x^2$ must have an even number of factors of $11$, $y$ will have an odd number of factors of $11$.  And $y$ will have an even number of each of its other prime factors.  
\end{hint}
\end{freeResponse}
\end{problem}

\begin{problem}
Suppose $x$ and $y$ are integers.  If $x^2 = 25\cdot y$, what can you say about $y$? Explain your
  reasoning.
\begin{freeResponse}
\begin{hint}
Because $x^2$, $25y^2$, and $25$ are all perfect squares, $y$ must also be a perfect square 
\end{hint}
\end{freeResponse}
\end{problem}

%\begin{problem}
%When asked how many people were staying at the \textit{Hotel
%  Chevalier}, the clerk responded ``The number you seek is the
%  smallest positive integer such that dividing by $2$ yields a perfect
%  square, and dividing by $3$ yields a perfect cube.'' How many people
%  are staying at the hotel? Explain your reasoning.
%\end{problem}



\end{document}