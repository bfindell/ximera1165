
\documentclass[nooutcomes]{ximera}
%\documentclass[space,handout,nooutcomes]{ximera}

% For preamble materials

\usepackage{pgf,tikz}
\usepackage{mathrsfs}
\usetikzlibrary{arrows}
\usepackage{framed}
\usepackage{amsmath}
%\pgfplotsset{compat=1.16}

\graphicspath{
  {./}
  {algorithms/}
  {../algorithms/}
}

\pdfOnly{\renewenvironment{image}[1][]{\begin{center}}{\end{center}}}

%%% This set of code is all of our user defined commands
\newcommand{\bysame}{\mbox{\rule{3em}{.4pt}}\,}
\newcommand{\N}{\mathbb N}
\newcommand{\C}{\mathbb C}
\newcommand{\W}{\mathbb W}
\newcommand{\Z}{\mathbb Z}
\newcommand{\Q}{\mathbb Q}
\newcommand{\R}{\mathbb R}
\newcommand{\A}{\mathbb A}
\newcommand{\D}{\mathcal D}
\newcommand{\F}{\mathcal F}
\newcommand{\ph}{\varphi}
\newcommand{\ep}{\varepsilon}
\newcommand{\aph}{\alpha}
\newcommand{\QM}{\begin{center}{\huge\textbf{?}}\end{center}}

\renewcommand{\le}{\leqslant}
\renewcommand{\ge}{\geqslant}
\renewcommand{\a}{\wedge}
\renewcommand{\v}{\vee}
\renewcommand{\l}{\ell}
\newcommand{\mat}{\mathsf}
\renewcommand{\vec}{\mathbf}
\renewcommand{\subset}{\subseteq}
\renewcommand{\supset}{\supseteq}
\renewcommand{\emptyset}{\varnothing}
\newcommand{\xto}{\xrightarrow}
\renewcommand{\qedsymbol}{$\blacksquare$}
\newcommand{\bibname}{References and Further Reading}
\renewcommand{\bar}{\protect\overline}
\renewcommand{\hat}{\protect\widehat}
\renewcommand{\tilde}{\widetilde}
\newcommand{\tri}{\triangle}
\newcommand{\minipad}{\vspace{1ex}}
\newcommand{\leftexp}[2]{{\vphantom{#2}}^{#1}{#2}}

%% More user defined commands
\renewcommand{\epsilon}{\varepsilon}
\renewcommand{\theta}{\vartheta} %% only for kmath
\renewcommand{\l}{\ell}
\renewcommand{\d}{\, d}
\newcommand{\ddx}{\frac{d}{dx}}
\newcommand{\dydx}{\frac{dy}{dx}}


\usepackage{bigstrut}


\newenvironment{sectionOutcomes}{}{}

\usepackage{array}
%\setlength{\extrarowheight}{-.2cm}   % Commented out by Findell to fix table headings.  Was this for typesetting division?  
\newdimen\digitwidth
\settowidth\digitwidth{9}
\def~{\hspace{\digitwidth}}
\def\divrule#1#2{
\noalign{\moveright#1\digitwidth
\vbox{\hrule width#2\digitwidth}}}


\title{Geometric Sequences}
\author{Bart Snapp and Brad Findell and Jenny Sheldon}
\begin{document}
\begin{abstract}
Problems about geometric (and other) sequences.
\end{abstract}
\maketitle


%\begin{problem}
%Problem
%\begin{freeResponse}
%\begin{hint}
%Hint
%\end{hint}
%\end{freeResponse}
%\end{problem} 



\begin{problem}
A sequence has first two terms $1, 2, \dots$.  What type of sequence is this?
\begin{multipleChoice}
\choice{An arithmetic sequence.}
\choice{A geometric sequence.}
\choice{A quadratic sequence.}
\choice[correct]{It is impossible to tell.}
\end{multipleChoice}
\begin{hint}
	If we do not know the rule for generating the terms of this sequence, can we be sure we know the next term?
\end{hint}
\end{problem}



\begin{problem}
Sylvie has a bank account which contains $\$23$ currently.  She decides to set up a savings plan for depositing money in her account.  The first week, she will deposit $\$4$ into this account, and then each week afterwards, she will deposit $10\%$ more than she did the previous week.  

Would we use a geometric sequence to describe the amount she deposits each week?
\begin{multipleChoice}
\choice[correct]{Yes, this is a geometric sequence.}
\choice{No, this is not a geometric sequence.}
\end{multipleChoice}

\end{problem}



\begin{problem}
Sylvie has a bank account which contains $\$23$ currently.  She decides to set up a savings plan for depositing money in her account.  The first week, she will deposit $\$4$ into this account, and then each week afterwards, she will deposit $10\%$ more than she did the previous week.  

Fill out the following table corresponding to the amount Sylvie deposits in her bank account $n$ weeks from now. (Enter the exact amount; do not round to the nearest cent.)
\[
\begin{array}{c|c} \hline
n & \text{Deposit Amount} \\ \hline
0 & \answer{4}\\
1 & \answer{4.4} \\
2 & \answer{4.84} \\
3 & \answer{5.324}\\
4 & \answer{5.8564}\\
5 & \answer{6.44204}\\
6 & \answer{7.086244}\\
7 & \answer{7.7948684}\\
8 & \answer{8.57435524}\\
\end{array}
\]


\end{problem}



\begin{problem}
Sylvie has a bank account which contains $\$23$ currently.  She decides to set up a savings plan for depositing money in her account.  The first week, she will deposit $\$4$ into this account, and then each week afterwards, she will deposit $10\%$ more than she did the previous week.  


Write a recursive formula for the amount Sylvie deposits in her account in week $n$, using ``Next'' and ``Now'' to describe the situation.

\begin{prompt}
Sylvie will deposit \wordChoice{\choice[correct]{Next} \choice{Now} \choice{$n$} \choice{4} \choice{1.1} \choice{0.1}} $=$ \wordChoice{\choice{Next} \choice[correct]{Now} \choice{$n$} \choice{4} \choice{1.1} \choice{0.1}}  $\times$  \wordChoice{\choice{Next} \choice{Now} \choice{$n$} \choice{4} \choice[correct]{1.1} \choice{0.1}}.
\end{prompt}

\end{problem}




\begin{problem}
Sylvie has a bank account which contains $\$23$ currently.  She decides to set up a savings plan for depositing money in her account.  The first week, she will deposit $\$4$ into this account, and then each week afterwards, she will deposit $10\%$ more than she did the previous week.  

Write an explicit formula for $f(n)$, the amount she deposits in week $n$.

\begin{prompt}
Sylvie will deposit $f(n) = \answer[given]{4(1.1)^n}$ in week $n$.
\end{prompt}

\end{problem}




\begin{problem}
Seth is in a lab, measuring the amount of a decaying substance.  He knows that each day, he expects to have $\frac{1}{12}$ of the amount he had the previous day.  If he began with $90$ grams of the substance, predict how many grams he will have in the future.

How much of the substance will Seth expect to have on day $5$?
\begin{prompt}
Seth expects to have $\answer[given]{90(11/12)^{5}}$ grams of the substance.
\end{prompt}

\end{problem}



\begin{problem}
Seth is in a lab, measuring the amount of a decaying substance.  He knows that each day, he expects to have $\frac{11}{12}$ of the amount he had the previous day.  If he began with $90$ grams of the substance, predict how many grams he will have in the future.

When will Seth expect to have $75$ grams of the substance?
\begin{prompt}
Seth expects to have $75$ grams of the substance sometime between day $\answer[given]{2}$ and day $\answer[given]{3}$.
\end{prompt}

\end{problem}




\begin{problem}
Seth is in a lab, measuring the amount of a decaying substance.  He knows that each day, he expects to have $\frac{11}{12}$ of the amount he had the previous day.  If he began with $90$ grams of the substance, predict how many grams he will have in the future.

Write a recursive function for the amount of the substance Seth expects to have on day $n$.  Use $f(n-1)$ in your equation.
\begin{prompt}
Seth expects to have $f(n) = \answer[given]{f(n-1)(11/12)}$ grams of the substance.
\end{prompt}

\end{problem}




\begin{problem}
Seth is in a lab, measuring the amount of a decaying substance.  He knows that each day, he expects to have $\frac{11}{12}$ of the amount he had the previous day.  If he began with $90$ grams of the substance, predict how many grams he will have in the future.

Write an explicit function for the amount of the substance Seth expects to have on day $n$.
\begin{prompt}
Seth expects to have $f(n) = \answer[given]{90(11/12)^n}$ grams of the substance.
\end{prompt}

\end{problem}



\begin{problem}
Consider a sequence whose first term is $-2$, whose second term is $1$, and which has a constant second difference $\Delta \Delta (n) =3$.

What type of sequence is this?
\begin{multipleChoice}
\choice{Arithmetic.}
\choice{Geometric.}
\choice[correct]{Quadratic.}
\choice{Something else.}
\end{multipleChoice}


\end{problem}



\begin{problem}
Consider a sequence whose first term is $-2$, whose second term is $1$, and which has a constant second difference $\Delta \Delta (n) =3$.

Fill out the following table of values for this sequence.  Then, find explicit formulas for both $\Delta(n)$ and $f(n)$.

\[
\begin{array}{c|c|c|c} \hline
n & f(n) & \Delta(n) & \Delta \Delta(n) \\ \hline
1 & -2 & \answer{1} & \answer{3} \\
2 & -1 & \answer{4} & \answer{3} \\
3 & \answer{3} & \answer{7} & \answer{3} \\
4 & \answer{10} & \answer{10} & \answer{3} \\
5 & \answer{20} & \answer{13} & \answer{3} \\
6 & \answer{33} & \answer{16} & \answer{3} \\
\end{array}
\]

\begin{prompt}
We find that $\Delta(n) = \answer[given]{1+3(n-1)}$ and that $f(n) = \answer[given]{-2 + \frac{3}{2} (n)(n-1) - 2(n-1)}$.
\end{prompt}

\end{problem}




\begin{problem}
Assume the sequence below is a geometric sequence.  Fill in the blanks.

\[
\answer{3}, 6, 12, \answer{24}, \answer{48}, \answer{96}, \answer{192}, \dots
\]

\end{problem}




\begin{problem}
Assume the sequence below is a geometric sequence.  Fill in the blanks.

\[
\dots, \answer{-4}, 6, \answer{-9}, \answer{13.5}, -20.25, \answer{30.375}, \answer{-45.5625}, \dots
\]

\end{problem}




\begin{problem}
Assume the sequence below is a geometric sequence.  Fill in the blanks.

\[
\dots, \answer{16(0.75)^{(-0.5)}}, \answer{16(0.75)^{(-0.25)}}, 16, \answer{16(0.75)^{(0.25)}}, \answer{16(0.75)^{(0.5)}}, \answer{16(0.75)^{(0.75)}}, 12, \dots
\]

\end{problem}





\end{document}



