
\documentclass[nooutcomes]{ximera}
%\documentclass[space,handout,nooutcomes]{ximera}

% For preamble materials

\usepackage{pgf,tikz}
\usepackage{mathrsfs}
\usetikzlibrary{arrows}
\usepackage{framed}
\usepackage{amsmath}
%\pgfplotsset{compat=1.16}

\graphicspath{
  {./}
  {algorithms/}
  {../algorithms/}
}

\pdfOnly{\renewenvironment{image}[1][]{\begin{center}}{\end{center}}}

%%% This set of code is all of our user defined commands
\newcommand{\bysame}{\mbox{\rule{3em}{.4pt}}\,}
\newcommand{\N}{\mathbb N}
\newcommand{\C}{\mathbb C}
\newcommand{\W}{\mathbb W}
\newcommand{\Z}{\mathbb Z}
\newcommand{\Q}{\mathbb Q}
\newcommand{\R}{\mathbb R}
\newcommand{\A}{\mathbb A}
\newcommand{\D}{\mathcal D}
\newcommand{\F}{\mathcal F}
\newcommand{\ph}{\varphi}
\newcommand{\ep}{\varepsilon}
\newcommand{\aph}{\alpha}
\newcommand{\QM}{\begin{center}{\huge\textbf{?}}\end{center}}

\renewcommand{\le}{\leqslant}
\renewcommand{\ge}{\geqslant}
\renewcommand{\a}{\wedge}
\renewcommand{\v}{\vee}
\renewcommand{\l}{\ell}
\newcommand{\mat}{\mathsf}
\renewcommand{\vec}{\mathbf}
\renewcommand{\subset}{\subseteq}
\renewcommand{\supset}{\supseteq}
\renewcommand{\emptyset}{\varnothing}
\newcommand{\xto}{\xrightarrow}
\renewcommand{\qedsymbol}{$\blacksquare$}
\newcommand{\bibname}{References and Further Reading}
\renewcommand{\bar}{\protect\overline}
\renewcommand{\hat}{\protect\widehat}
\renewcommand{\tilde}{\widetilde}
\newcommand{\tri}{\triangle}
\newcommand{\minipad}{\vspace{1ex}}
\newcommand{\leftexp}[2]{{\vphantom{#2}}^{#1}{#2}}

%% More user defined commands
\renewcommand{\epsilon}{\varepsilon}
\renewcommand{\theta}{\vartheta} %% only for kmath
\renewcommand{\l}{\ell}
\renewcommand{\d}{\, d}
\newcommand{\ddx}{\frac{d}{dx}}
\newcommand{\dydx}{\frac{dy}{dx}}


\usepackage{bigstrut}


\newenvironment{sectionOutcomes}{}{}

\usepackage{array}
%\setlength{\extrarowheight}{-.2cm}   % Commented out by Findell to fix table headings.  Was this for typesetting division?  
\newdimen\digitwidth
\settowidth\digitwidth{9}
\def~{\hspace{\digitwidth}}
\def\divrule#1#2{
\noalign{\moveright#1\digitwidth
\vbox{\hrule width#2\digitwidth}}}


\title{Arithmetic Sequences}
\author{Bart Snapp and Brad Findell and Jenny Sheldon}
\begin{document}
\begin{abstract}
Problems about arithmetic sequences.
\end{abstract}
\maketitle


%\begin{problem}
%Problem
%\begin{freeResponse}
%\begin{hint}
%Hint
%\end{hint}
%\end{freeResponse}
%\end{problem} 


\begin{problem}
Trish has a bank account which has $\$23$ currently.  She decides to set up a savings plan in which each week she will deposit $\$2$ into this account. (Sadly, it does not earn interest.)

Would we use an arithmetic sequence to describe the balance in her account each week?
\begin{multipleChoice}
\choice[correct]{Yes, this is an arithmetic sequence.}
\choice{No, this is not an arithmetic sequence.}
\end{multipleChoice}

\begin{problem}
Each Saturday, Tabitha goes to her local farmer's market.  Last week, she bought 4 ounces of almond flour, and this week she decided she needed twice as much almond flour as she did last week.  Quickly, she develops a habit of buying twice as much almond flour as the previous week each time she goes to the farmer's market.

Would we use an arithmetic sequence to describe this situation?
\begin{multipleChoice}
\choice{Yes, this is an arithmetic sequence.}
\choice[correct]{No, this is not an arithmetic sequence.}
\end{multipleChoice}
\end{problem}
\end{problem}



\begin{problem}
Trish has a bank account which has $\$23$ currently.  She decides to set up a savings plan in which each week she will deposit $\$2$ into this account. (Sadly, it does not earn interest.)

Fill out the following table corresponding to the amount Trish has in her bank account $n$ weeks from now.
\[
\begin{array}{c|c} \hline
n & \text{Account Balance} \\ \hline
1 & \answer{25} \\
2 & \answer{27} \\
3 & \answer{29}\\
4 & \answer{31}\\
5 & \answer{33}\\
6 & \answer{35}\\
7 & \answer{37}\\
8 & \answer{39}\\
\end{array}
\]
\end{problem}



\begin{problem}
Trish has a bank account which has $\$23$ currently.  She decides to set up a savings plan in which each week she will deposit $\$2$ into this account. (Sadly, it does not earn interest.)

How much money does Trish have in her bank account after $52$ weeks have passed?

\begin{prompt}
Trish has $\answer[given]{127}$ in her bank account.
\end{prompt}
\end{problem}




\begin{problem}
Trish has a bank account which has $\$23$ currently.  She decides to set up a savings plan in which each week she will deposit $\$2$ into this account. (Sadly, it does not earn interest.)

Write an expression for the amount of money Trish will have in her bank account $n$ weeks from now.

\begin{prompt}
Trish will have $\answer[given]{23+2n}$ in her bank account.
\end{prompt}
\end{problem}





\begin{problem}
Terry is starting a small business.  In year $2$, they have $325$ customers, and based on previous growth, their goal is to increase the number of customers by $230$ each year.


How many customers does Terry hope to have in year $5$?

\begin{prompt}
Terry hopes to have $\answer[given]{1015}$ customers in year $5$.
\end{prompt}
\end{problem}




\begin{problem}
Terry is starting a small business.  In year $2$, they have $325$ customers, and based on previous growth, their goal is to increase the number of customers by $230$ each year.


Write a recursive formula for the number of customers Terry has in year $n$, using ``Next'' and ``Now'' to describe the situation.

\begin{prompt}
Terry hopes to have \wordChoice{\choice[correct]{Next} \choice{Now} \choice{$n$} \choice{230} \choice{325}} $=$ \wordChoice{\choice{Next} \choice[correct]{Now} \choice{$n$} \choice{230} \choice{325}} $+$ \wordChoice{\choice{Next} \choice{Now} \choice{$n$} \choice[correct]{230} \choice{325}}.
\end{prompt}
\end{problem}



\begin{problem}
Terry is starting a small business.  In year $2$, they have $325$ customers, and based on previous growth, their goal is to increase the number of customers by $230$ each year.


Write a recursive formula for the number of customers Terry has in year $n$, using ``$f(n+1)$'' and ``$f(n)$'' to describe the situation.

\begin{prompt}
Terry hopes to have \wordChoice{\choice[correct]{$f(n+1)$} \choice{$f(n)$} \choice{$n$} \choice{230} \choice{325}} $=$ \wordChoice{\choice{$f(n+1)$} \choice[correct]{$f(n)$} \choice{$n$} \choice{230} \choice{325}} $+$ \wordChoice{\choice{$f(n+1)$} \choice{$f(n)$} \choice{$n$} \choice[correct]{230} \choice{325}}.
\end{prompt}
\end{problem}





\begin{problem}
Tripp is training for a bike ride across the state.  In week 8 of his training, he rode $32.2$ miles.  Each week, his trainer has recommended that he ride $6.3$ miles more than he did the previous week.

How far will Tripp ride in week 12 of his training?

\begin{prompt}
Tripp will ride $\answer[given]{57.4}$ miles in week 12.
\end{prompt}
\end{problem}



\begin{problem}
Tripp is training for a bike ride across the state.  In week 8 of his training, he rode $32.2$ miles.  Each week, his trainer has recommended that he ride $6.3$ miles more than he did the previous week.

Write a recursive function for the number of miles $f(n)$ Tripp will ride in week $n$ of his training.  Use $f(n-1)$ in your equation.

\begin{prompt}
Tripp will ride $f(n) = \answer[given]{f(n-1) + 6.3}$ miles in week $n$.
\end{prompt}
\end{problem}



\begin{problem}
Tripp is training for a bike ride across the state.  In week 8 of his training, he rode $32.2$ miles.  Each week, his trainer has recommended that he ride $6.3$ miles more than he did the previous week.

Write an explicit function for the number of miles $f(n)$ Tripp will ride in week $n$ of his training.

\begin{prompt}
Tripp will ride $f(n) = \answer[given]{6.3n-18.2}$ miles in week $n$.
\end{prompt}
\end{problem}




\begin{problem}
Tripp is training for a bike ride across the state.  In week 8 of his training, he rode $32.2$ miles.  Each week, his trainer has recommended that he ride $6.3$ miles more than he did the previous week.

During what week will Tripp first ride more than $200$ miles?

\begin{prompt}
Tripp will ride more than $200$ miles beginning in week $\answer[given]{35}$.
\end{prompt}
\end{problem}





\begin{problem}
Therese is monitoring her neighborhood swimming pool as it fills with water at a constant rate of $16$ gallons per minute.  After $20$ minutes, the pool has $454$ gallons.  How much water did the pool have before Therese began filling the pool?

\begin{prompt}
The pool had $\answer[given]{134}$ gallons when Therese began.
\end{prompt}
\end{problem}




\begin{problem}
Therese is monitoring her neighborhood swimming pool as it fills with water at a constant rate of $16$ gallons per minute.  After $20$ minutes, the pool has $454$ gallons.  Write an equation for the number of gallons in the pool after $n$ minutes.

\begin{prompt}
The pool has $f(n) = \answer[given]{134+16n}$ gallons after $n$ minutes.
\end{prompt}
\end{problem}




\begin{problem}
Assume the sequence below is an arithmetic sequence.  Fill in the blanks.

\[
\answer{-5}, -1, 3, \answer{7}, \answer{11}, \answer{15}, \answer{19}, \dots
\]

\end{problem}





\begin{problem}
Assume the sequence below is an arithmetic sequence.  Fill in the blanks.

\[
\dots, \answer{-5}, -10, \answer{-15}, \answer{-20}, \answer{-25}, -30, \answer{-35}, \dots
\]

\end{problem}




\begin{problem}
Assume the sequence below is an arithmetic sequence.  Fill in the blanks.

\[
\dots, \answer{2.7}, \answer{3.3}, 3.9, \answer{4.5}, \answer{5.1}, 5.7, \answer{6.3}, \dots
\]

\end{problem}




\begin{problem}
Imagine drawing a graph like the one below to represent each of the situations with Trish, Terry, Tripp, and Therese.

\begin{center}
\begin{tikzpicture}
\draw[->, thick] (-1, 0)--(8,0);
\draw[->, thick] (0,-1)--(0, 8);
\foreach \x in {1, 2, 3, 4, 5, 6} \draw[fill=black] (\x, \x+1) circle (2pt);
\end{tikzpicture}
\end{center}

In which of the situations does it make sense to connect the dots with a line?
\begin{selectAll}
\choice{Trish and her bank account.}
\choice{Terry and their small business.}
\choice{Tripp and his bike travel.}
\choice[correct]{Therese and the swimming pool.}
\end{selectAll}




\end{problem}




\end{document}



