
\documentclass[nooutcomes]{ximera}
%\documentclass[space,handout,nooutcomes]{ximera}

% For preamble materials

\usepackage{pgf,tikz}
\usepackage{mathrsfs}
\usetikzlibrary{arrows}
\usepackage{framed}
\usepackage{amsmath}
%\pgfplotsset{compat=1.16}

\graphicspath{
  {./}
  {algorithms/}
  {../algorithms/}
}

\pdfOnly{\renewenvironment{image}[1][]{\begin{center}}{\end{center}}}

%%% This set of code is all of our user defined commands
\newcommand{\bysame}{\mbox{\rule{3em}{.4pt}}\,}
\newcommand{\N}{\mathbb N}
\newcommand{\C}{\mathbb C}
\newcommand{\W}{\mathbb W}
\newcommand{\Z}{\mathbb Z}
\newcommand{\Q}{\mathbb Q}
\newcommand{\R}{\mathbb R}
\newcommand{\A}{\mathbb A}
\newcommand{\D}{\mathcal D}
\newcommand{\F}{\mathcal F}
\newcommand{\ph}{\varphi}
\newcommand{\ep}{\varepsilon}
\newcommand{\aph}{\alpha}
\newcommand{\QM}{\begin{center}{\huge\textbf{?}}\end{center}}

\renewcommand{\le}{\leqslant}
\renewcommand{\ge}{\geqslant}
\renewcommand{\a}{\wedge}
\renewcommand{\v}{\vee}
\renewcommand{\l}{\ell}
\newcommand{\mat}{\mathsf}
\renewcommand{\vec}{\mathbf}
\renewcommand{\subset}{\subseteq}
\renewcommand{\supset}{\supseteq}
\renewcommand{\emptyset}{\varnothing}
\newcommand{\xto}{\xrightarrow}
\renewcommand{\qedsymbol}{$\blacksquare$}
\newcommand{\bibname}{References and Further Reading}
\renewcommand{\bar}{\protect\overline}
\renewcommand{\hat}{\protect\widehat}
\renewcommand{\tilde}{\widetilde}
\newcommand{\tri}{\triangle}
\newcommand{\minipad}{\vspace{1ex}}
\newcommand{\leftexp}[2]{{\vphantom{#2}}^{#1}{#2}}

%% More user defined commands
\renewcommand{\epsilon}{\varepsilon}
\renewcommand{\theta}{\vartheta} %% only for kmath
\renewcommand{\l}{\ell}
\renewcommand{\d}{\, d}
\newcommand{\ddx}{\frac{d}{dx}}
\newcommand{\dydx}{\frac{dy}{dx}}


\usepackage{bigstrut}


\newenvironment{sectionOutcomes}{}{}

\usepackage{array}
%\setlength{\extrarowheight}{-.2cm}   % Commented out by Findell to fix table headings.  Was this for typesetting division?  
\newdimen\digitwidth
\settowidth\digitwidth{9}
\def~{\hspace{\digitwidth}}
\def\divrule#1#2{
\noalign{\moveright#1\digitwidth
\vbox{\hrule width#2\digitwidth}}}


\title{Numbers}
\author{Bart Snapp and Brad Findell and Jenny Sheldon}
\begin{document}
\begin{abstract}
Numbers you should know. 
\end{abstract}
\maketitle


%\begin{problem}
%Problem
%\begin{freeResponse}
%\begin{hint}
%Hint
%\end{hint}
%\end{freeResponse}
%\end{problem} 


We assume you know your multiplication facts up to $12\times 12$. Because you are math teachers, it helps to have some other numbers on the tip of your tongue---or at least be able to figure them out quickly in your head. 


\begin{problem}
Cubes:
\begin{enumerate}
\item $2^3 = \answer{8}$
\item $3^3 = \answer{27}$
\item $4^3 = \answer{64}$
\item $5^3 = \answer{125}$
\item $6^3 = \answer{216}$
\item $7^3 = \answer{343}$
\item $8^3 = \answer{512}$
\item $9^3 = \answer{729}$
\item $10^3 = \answer{1000}$
\item $11^3 = \answer{1331}$
\end{enumerate}
\end{problem}

\begin{problem}
Factorials: $4!$ means $4\cdot3\cdot2\cdot1$.  
\begin{enumerate}
\item $4!= \answer{24}$
\item $5!= \answer{120}$
\item $6!= \answer{720}$
\item $7!= \answer{5040}$
\item $3!= \answer{6}$
\item $2!= \answer{2}$
\item $1!= \answer{1}$
\item $0!= \answer{1}$
\end{enumerate}
\end{problem}

\begin{problem}
Powers of 2
\begin{enumerate}
\item $2^4 = \answer{16}$
\item $2^5 = \answer{32}$
\item $2^6 = \answer{64}$
\item $2^7 = \answer{128}$
\item $2^8 = \answer{256}$
\item $2^9 = \answer{512}$
\item $2^10 = \answer{1024}$
\end{enumerate}
\end{problem}

\begin{problem}
Because computers are designed around base $\answer[format=string]{two}$, the powers of two turn out to be very important.  In particular, because $2^{10}=\answer{1024}$, which is approximately $1000$, computer science borrowed the base-ten prefixes from the metric system.  

Computer memory and storage is typically measured in bytes, where 1 byte can store one ``character,'' such as a letter, punctuation mark, or space in this sentence.  With the metric prefixes, we have, for example, 
\begin{enumerate}
\item $1$ Kilobyte (1 KB) $=  2^{10}\textrm{ bytes }\approx \answer{10^3}$ bytes = one \wordChoice{\choice[correct]{thousand} \choice{million} \choice{billion} \choice{trillion}} bytes.
\item $1$ Megabyte (1 MB) $=  \answer{2^{20}} \textrm{ bytes } \approx \answer{10^6}$ bytes = one \wordChoice{\choice{thousand} \choice[correct]{million} \choice{billion} \choice{trillion}} bytes.
\item $1$ Gigabyte (1 GB) $=  \answer{2^{30}} \textrm{ bytes } \approx \answer{10^9}$ bytes = one \wordChoice{\choice{thousand} \choice{million} \choice[correct]{billion} \choice{trillion}} bytes.
\item $1$ Terabyte (1 TB) $=  \answer{2^{40}} \textrm{ bytes } \approx \answer{10^{12}}$ bytes = one \wordChoice{\choice{thousand} \choice{million} \choice{billion} \choice[correct]{trillion}} bytes.
\end{enumerate}


And 1 byte is composed of 8 bits, each of which is a base-two digit, either 0 or 1.  Internet speeds are usually given in Mb/sec, which means ``Megabits per second.'' So if your Internet speed is $100$ Mb/sec, 
that means $\answer{100\cdot10^6}$ bits per second.  (Type your answer without commas.)

Computer processor speeds are typically given in GHz, where 1 Hz = 1 cycle per second.  Many processors operations require 1 cycle to complete, so we imagine that a 2.4 GHz processor completes $\answer{2.4\cdot10^9}$ operations per second.  

Many modern processors have several `cores' which can perform operations at the same time.  But never mind.  

In 1984, a fully-equipped IBM PC/XT had $640$ KB of memory, a $10$ MB hard drive, and processor running at $4.77$ MHz.  In those years, it was possible to connect others through an ``online service provider'' (the World Wide Web did not exist), with a modem operating at 300 bits per second, tying up the phone line, and at a cost of $\$6$---or even as much as $\$30$ per hour.  (A leading online service provider was Compuserve, based in Columbus, Ohio.)  

\begin{enumerate}
\item Processor speed: $\answer{2.4\cdot10^9/4.77\cdot10^6}$
\item Disk storage: $\answer{512\cdot10^9/10\cdot10^6}$
\item Memory: $\answer{16\cdot10^9/640*10^3}$
\item Online speed: $\answer[tolerance=5000]{100\cdot10^6/300}$. 
\end{enumerate}
\end{problem}

\end{document}



