
\documentclass[nooutcomes]{ximera}
%\documentclass[space,handout,nooutcomes]{ximera}

% For preamble materials

\usepackage{pgf,tikz}
\usepackage{mathrsfs}
\usetikzlibrary{arrows}
\usepackage{framed}
\usepackage{amsmath}
%\pgfplotsset{compat=1.16}

\graphicspath{
  {./}
  {algorithms/}
  {../algorithms/}
}

\pdfOnly{\renewenvironment{image}[1][]{\begin{center}}{\end{center}}}

%%% This set of code is all of our user defined commands
\newcommand{\bysame}{\mbox{\rule{3em}{.4pt}}\,}
\newcommand{\N}{\mathbb N}
\newcommand{\C}{\mathbb C}
\newcommand{\W}{\mathbb W}
\newcommand{\Z}{\mathbb Z}
\newcommand{\Q}{\mathbb Q}
\newcommand{\R}{\mathbb R}
\newcommand{\A}{\mathbb A}
\newcommand{\D}{\mathcal D}
\newcommand{\F}{\mathcal F}
\newcommand{\ph}{\varphi}
\newcommand{\ep}{\varepsilon}
\newcommand{\aph}{\alpha}
\newcommand{\QM}{\begin{center}{\huge\textbf{?}}\end{center}}

\renewcommand{\le}{\leqslant}
\renewcommand{\ge}{\geqslant}
\renewcommand{\a}{\wedge}
\renewcommand{\v}{\vee}
\renewcommand{\l}{\ell}
\newcommand{\mat}{\mathsf}
\renewcommand{\vec}{\mathbf}
\renewcommand{\subset}{\subseteq}
\renewcommand{\supset}{\supseteq}
\renewcommand{\emptyset}{\varnothing}
\newcommand{\xto}{\xrightarrow}
\renewcommand{\qedsymbol}{$\blacksquare$}
\newcommand{\bibname}{References and Further Reading}
\renewcommand{\bar}{\protect\overline}
\renewcommand{\hat}{\protect\widehat}
\renewcommand{\tilde}{\widetilde}
\newcommand{\tri}{\triangle}
\newcommand{\minipad}{\vspace{1ex}}
\newcommand{\leftexp}[2]{{\vphantom{#2}}^{#1}{#2}}

%% More user defined commands
\renewcommand{\epsilon}{\varepsilon}
\renewcommand{\theta}{\vartheta} %% only for kmath
\renewcommand{\l}{\ell}
\renewcommand{\d}{\, d}
\newcommand{\ddx}{\frac{d}{dx}}
\newcommand{\dydx}{\frac{dy}{dx}}


\usepackage{bigstrut}


\newenvironment{sectionOutcomes}{}{}

\usepackage{array}
%\setlength{\extrarowheight}{-.2cm}   % Commented out by Findell to fix table headings.  Was this for typesetting division?  
\newdimen\digitwidth
\settowidth\digitwidth{9}
\def~{\hspace{\digitwidth}}
\def\divrule#1#2{
\noalign{\moveright#1\digitwidth
\vbox{\hrule width#2\digitwidth}}}


\title{Precision}
\author{Bart Snapp and Brad Findell and Jenny Sheldon}
\begin{document}
\begin{abstract}
Important points about being precise in mathematical writing.
\end{abstract}
\maketitle


%\begin{problem}
%Problem
%\begin{freeResponse}
%\begin{hint}
%Hint
%\end{hint}
%\end{freeResponse}
%\end{problem} 
%
% $\answer[format=string]{}$


%Rules of exponents
%Instead of cross multiplying 
%Unit rates
%Use of variables 
%Use of equals sign
%Decimals
%
%When you write by hand, distinguish $+$ from $t$, $2$ from $z$, and $7$ from $1$. 
%
%For number systems, use ``blackboard bold.''  
%
%Write congruences with built-in correspondence. 
%Angles, segments, and their measures.  
%
%Use an equation editor or ?math mode? for typesetting mathematics.  Variables should appear as italic with serifs. 
%
%To avoid ambiguity in calculation, it is usually better to write fractions and division with a horizontal bar rather with a slash or with the division symbol.  
%
%Dots
%
%Avoid mnemonics. 
%Convention
%Cancelling
%Use of equals sign. 
%
%Spelling: Commutative, Indeterminate, Asymptote, Isosceles
%

\begin{problem}
In mathematical writing, we aim to use vocabulary, grammar, and notation \wordChoice{\choice{happily} \choice{vaguely} \choice[correct]{precisely} \choice{quickly}}, meaning with careful attention to details.  

The following problems are intended to help you do and write mathematics more precisely.  
\end{problem}


\begin{problem}
In mathematics, we distinguish expressions from equations and inequalities.  Equations and inequalities can be complete sentences.  Expressions are noun phrases that may be used within sentences.  For example: 

Let $n$ be the number of nickels and $d$ be the number of dimes.  Then the number of coins is $\answer{n+d}$, and their value (in cents) is $\answer{5n+10d}$.  
\end{problem}

\begin{problem}
\begin{enumerate}
\item The statement $a+b = b+a$ is called the $\answer[format=string]{commutative}$ property of $\answer[format=string]{addition}$.

\item The statement $(a+b)+c = a+(b+c)$ is called the $\answer[format=string]{associative}$ property of $\answer[format=string]{addition}$.

\item The statement $ab = ba$ is called the $\answer[format=string]{commutative}$ property of $\answer[format=string]{multiplication}$.

\item The statement $(ab)c = a(bc)$ is called the $\answer[format=string]{associative}$ property of $\answer[format=string]{multiplication}$. 

These equations are called \wordChoice{\choice{functions} \choice{expressions} \choice[correct]{identities} \choice{relations}} because they are true for \textbf{all} values of the variables.  
\end{enumerate}
Notice that each of these properties involves \textbf{one operation}: either addition or multiplication, but not both.  
\end{problem}

\begin{problem}
Which of the following statements are identities (i.e., always true)? 
\begin{selectAll}
\choice{$(a+b)^2=a^2+b^2$}
\choice[correct]{$3(x+2)=3x+6$}
\choice{$m(pq)=(mp)(mq)$}
\choice[correct]{$2x+3x=(2+3)x$}
\choice[correct]{$(3+x)(4+x)=12+7x+x^2$}
\choice{$\sqrt{x^2+y^2}=x+y$}
\choice[correct]{$(4x)^3=4^3 x^3$}
\end{selectAll}
\end{problem}

\begin{problem}
Which of the following are examples of the \textbf{distributive} property? 
\begin{selectAll}
\choice{$(a+b)^2=a^2+b^2$}
\choice[correct]{$3(x+2)=3x+6$}
\choice{$m(pq)=(mp)(mq)$}
\choice[correct]{$2x+3x=(2+3)x$}
\choice[correct]{$(3+x)(4+x)=12+7x+x^2$}
\choice{$\sqrt{x^2+y^2}=x+y$}
\choice{$(4x)^3=4^3 x^3$}
\end{selectAll}
\end{problem}



\begin{problem}
The statement $a(b+c)=ab+ac$ is called the $\answer[format=string]{distributive}$ property of $\answer[format=string]{multiplication}$ over $\answer[format=string]{addition}$.'' 
\begin{feedback}
Correct.  We often abbreviate this as ``the distributive property,'' but it is important to keep in mind the full name because it shows how 
these \textbf{two operations} work together. 
\end{feedback}
\end{problem}

\begin{problem}
The expression simplification $3x+4x+y+5y=7x+6y$ is an example of a procedure often called ``collecting 
$\answer[format=string]{like terms}$'', but in fact is it yet another example of the 
$\answer[format=string]{distributive}$ property. 
\begin{problem}
Correct.  Here are some details: 
\[
3x+4x+y+5y=(3+4)\answer{x}+\left(\answer{1}+5\right)y=7x+6y
\]
\end{problem}
\end{problem}

\begin{problem}
A \textbf{rational number} is any number that can be expressed as $\answer{\frac{a}{b}}$ where $a$ and $b$ are $\answer[format=string]{integers}$ and $b$ \wordChoice{\choice{$=$} \choice{$<$} \choice{$>$} \choice[correct]{$\ne$}} $\answer{0}$.
\end{problem}

\begin{problem}
A \textbf{function} is a rule that assigns to each 
\wordChoice{\choice{number} \choice[correct]{input} \choice{output} \choice{formula}} exactly one \wordChoice{\choice{number} \choice{input} \choice[correct]{output} \choice{formula}}.  The set of 
all input values is called the $\answer[format=string]{domain}$ of the function.  The set of 
all output values is called the $\answer[format=string]{range}$ of the function.
\end{problem}

\begin{problem}
Consider the equation $A+b+2=2a+B+5$.  Solve the equation for $A$.  

$A=\answer{2a+B-b+3}$.  
\begin{hint}
Are upper- and lower-case letters interchangeable?  
\end{hint}
\begin{feedback}[correct]
Yes.  Always distinguish between upper- and lower-case letters.  Assume that $A\ne a$, $B\ne b$, etc.  Do not use them interchangeably.  
\end{feedback}
\end{problem}

\begin{problem}
Sandy is accepting orders for boxes of cookies, which are $\$4$ each plus $\$6$ for shipping.  As an example, he writes the calculation for an order of $5$ boxes as follows: 
\[
5\cdot 4 = 20 + 6 = 26.
\]
Has Sandy written this in a way that is mathematically precise?  
\begin{multipleChoice}
\choice{Yes}
\choice[correct]{No}
\choice{It depends}
\end{multipleChoice}
\begin{problem}
Sandy has misused the equals sign by writing $5\cdot 4 = \answer{20 + 6}$, which is false.  

Here are two possible ways to fix this: 
\begin{itemize}
\item One-sentence method: $5\cdot 4 + 6 = 20 + 6 = 26$.
\item Two-sentence method: $5\cdot 4 = 20$.  Then $20 + 6 = 26$.
\end{itemize}
\end{problem}
\end{problem}

\begin{problem}
In this class, we avoid ``cross multiplication,'' because the procedure is too often misused, and few students have any idea why it works.  Instead, we apply a general method for solving equations involving fractions: $\answer[format=string]{multiply}$ the equation by the $\answer[format=string]{denominators}$, either one at a time or all at once (if you are brave).  
\end{problem}

\begin{problem}
Is $1$ prime, composite, neither, or both?  
\begin{multipleChoice}
\choice{prime}
\choice{composite}
\choice[correct]{neither}
\choice{both}
\end{multipleChoice}

\begin{problem}
Correct!  We want the prime factorization of a number to be unique, up to the order of the factors.  Clearly $1$ is not composite.  If $1$ were prime, then the prime factorization of $12$ could be $2\cdot 2\cdot 3 \cdot 1\cdot 1$, or $2\cdot 2\cdot 3$ times as many $1$'s as you like.  Better to decide that $1$ is \emph{not} prime.  

With this decision, what can be said about the typical definition: ``A counting number is \emph{prime} if its only factors are $1$ and itself.''
\begin{multipleChoice}
\choice{It is correct.}
\choice[correct]{It is ambiguous and therefore not satisfactory.}
\choice{It is incorrect.}
\end{multipleChoice}
\begin{problem}
That's right.  A statement is \emph{ambiguous} if it can be interpreted in more than one way.  Definitions should be clear and unambiguous.

\begin{prompt}
Here is a better definition:  ``A counting number is \emph{prime} if it has \wordChoice{\choice{at least}\choice{at most}\choice[correct]{exactly}}
 $\answer{2}$ factors.''
\end{prompt}
\end{problem}
\end{problem}
\end{problem}


\begin{problem}
In math class, you might hear the following definition:  
\[
a^n\text{ means }a\text{ multiplied by itself }n\text{ times.}  
\]
Is the definition true or false?  
\begin{multipleChoice}
\choice{True}
\choice[correct]{False}
\end{multipleChoice}
\begin{problem}
Correct!  For example, in $3^5=3\cdot3\cdot3\cdot3\cdot3$, there are only $\answer{4}$ multiplications.  

Furthermore, when the exponent is negative or a fraction, it is hard to think about repeated multiplication.  

So we need a better definition.  Here is one possibility: 

When $n$ is a \wordChoice{\choice{real number} \choice{rational number} \choice[correct]{counting number} \choice{integer} }, $a^n$ means the $\answer[format=string]{product}$ of $\answer{n}$ (copies of) $\answer{a}$.
\end{problem}
\end{problem}



\end{document}



