
\documentclass[nooutcomes]{ximera}
%\documentclass[space,handout,nooutcomes]{ximera}

% For preamble materials

\usepackage{pgf,tikz}
\usepackage{mathrsfs}
\usetikzlibrary{arrows}
\usepackage{framed}
\usepackage{amsmath}
%\pgfplotsset{compat=1.16}

\graphicspath{
  {./}
  {algorithms/}
  {../algorithms/}
}

\pdfOnly{\renewenvironment{image}[1][]{\begin{center}}{\end{center}}}

%%% This set of code is all of our user defined commands
\newcommand{\bysame}{\mbox{\rule{3em}{.4pt}}\,}
\newcommand{\N}{\mathbb N}
\newcommand{\C}{\mathbb C}
\newcommand{\W}{\mathbb W}
\newcommand{\Z}{\mathbb Z}
\newcommand{\Q}{\mathbb Q}
\newcommand{\R}{\mathbb R}
\newcommand{\A}{\mathbb A}
\newcommand{\D}{\mathcal D}
\newcommand{\F}{\mathcal F}
\newcommand{\ph}{\varphi}
\newcommand{\ep}{\varepsilon}
\newcommand{\aph}{\alpha}
\newcommand{\QM}{\begin{center}{\huge\textbf{?}}\end{center}}

\renewcommand{\le}{\leqslant}
\renewcommand{\ge}{\geqslant}
\renewcommand{\a}{\wedge}
\renewcommand{\v}{\vee}
\renewcommand{\l}{\ell}
\newcommand{\mat}{\mathsf}
\renewcommand{\vec}{\mathbf}
\renewcommand{\subset}{\subseteq}
\renewcommand{\supset}{\supseteq}
\renewcommand{\emptyset}{\varnothing}
\newcommand{\xto}{\xrightarrow}
\renewcommand{\qedsymbol}{$\blacksquare$}
\newcommand{\bibname}{References and Further Reading}
\renewcommand{\bar}{\protect\overline}
\renewcommand{\hat}{\protect\widehat}
\renewcommand{\tilde}{\widetilde}
\newcommand{\tri}{\triangle}
\newcommand{\minipad}{\vspace{1ex}}
\newcommand{\leftexp}[2]{{\vphantom{#2}}^{#1}{#2}}

%% More user defined commands
\renewcommand{\epsilon}{\varepsilon}
\renewcommand{\theta}{\vartheta} %% only for kmath
\renewcommand{\l}{\ell}
\renewcommand{\d}{\, d}
\newcommand{\ddx}{\frac{d}{dx}}
\newcommand{\dydx}{\frac{dy}{dx}}


\usepackage{bigstrut}


\newenvironment{sectionOutcomes}{}{}

\usepackage{array}
%\setlength{\extrarowheight}{-.2cm}   % Commented out by Findell to fix table headings.  Was this for typesetting division?  
\newdimen\digitwidth
\settowidth\digitwidth{9}
\def~{\hspace{\digitwidth}}
\def\divrule#1#2{
\noalign{\moveright#1\digitwidth
\vbox{\hrule width#2\digitwidth}}}


\title{Exponential growth and decay}
\author{Brad Findell}
\begin{document}
\begin{abstract}
Percent increase and decrease; exponential growth and decay. % exponential functions, and the rules of exponents. 
\end{abstract}
\maketitle


%\begin{problem}
%Problem
%\begin{freeResponse}
%\begin{hint}
%Hint
%\end{hint}
%\end{freeResponse}
%\end{problem} 


%Percent growth or decay -> exponential growth
%	What is your answer? 
%	Naive way to do it (fill-in blank)
%	Sophisticated way to do it (fill-in blank)
%	Why does it work? Distributive property.  
%Explaining rules of exponents
%Exponential growth in context.  Where does your explicit formula work?  (0, negative, non-integer)
%Backwards in time


\begin{problem}
The population of Metroville is $380,\!000$ in January $2024$, and it has been growing at $3\%$ per year.  We say the \emph{growth rate} has been $3\%$.  

What will the population be January $2025$?  $\answer{380000\cdot 1.03}$.  
\begin{problem}
Correct!  

Some people make the computations as follows:  First, compute $3\%$ of the population, which is $\answer{0.03\cdot 380000}$.  Then add that to the original population.  

Other people use a more efficient method:  Just multiply $380,\!000$ by $\answer{1.03}$.  

\begin{problem}
Question: Why are these methods both correct?  In other words, why is  
\[
0.03\cdot 380,\!000 + 380,\!000 = 1.03 \cdot 380,\!000?  
\]
Answer: By the $\answer[format=string]{distributive}$ property, thinking of $380,\!000$ as a common $\answer[format=string]{factor}$. 

\begin{problem}
When thinking of growth in percentage terms, it helps to distinguish the \emph{growth rate}, in this case $3\%$, from the 
\emph{growth factor}, which is $1.03$.  And it often helps to think of the growth factor in percentage terms, for after 1 year, 
the population will be $103\%$ of the starting population.  

We can generalize these ideas as follows:  If a quantity $P$ is growing at a growth rate of $r$ (written as a decimal, interpreted as a percent), then after one unit of time, the 
new quantity will be $rP + P$, which can be factored as $\left(\answer{1+r}\right)P$.  In other words, when the growth rate is $r$, the growth factor is $\answer{1+r}$.  

\end{problem}
\end{problem}
\end{problem}
\end{problem}

\begin{problem}
Kayla purchased a new car for $\$16,\!000$ in January $2024$.  But once a car has been driven off the lot, it \emph{depreciates}, which is to say it loses value.  Kayla's accountant tells here that in normal economic times and with typical annual driving, her car will depreciate about $8\%$ per year.  

Using this rate of depreciation, what will the value of the car be in January $2025$?  $\answer{16000\cdot 0.92}$ dollars.  
\begin{problem}
Correct!  

Some people make the computations as follows:  First, compute $8\%$ of the purchase price, which is $\answer{0.08\cdot 16000}$ dollars.  Then subtract that from the purchase price.  

Other people use a more efficient method:  Just multiply $\$16,\!000$ by $\answer{0.92}$.  

\begin{problem}
Question: Why are these methods both correct?  In other words, why is  
\[
16,\!000 - 0.08\cdot 16,\!000 = 0.92 \cdot 16,\!000?  
\]
Answer: By the $\answer[format=string]{distributive}$ property, thinking of $16,\!000$ as a common $\answer[format=string]{factor}$. 

\begin{problem}
In finance, the opposite of depreciation is \emph{appreciation}, which means an increase in value.  In mathematics, appreciation and depreciation are examples of \emph{growth and decay}. 
 
When thinking of decay in percentage terms, it helps to distinguish the \emph{decay rate}, in this case $8\%$, from the 
\emph{decay factor}, which is $0.92$.  And it often helps to think of the decay factor in percentage terms, for after 1 year, 
the value of the car will be $92\%$ of the purchase price.  

We can generalize these ideas as follows:  If a quantity $P$ is decaying at a rate of $r$ (written as a decimal, interpreted as a percent), then after one unit of time, the 
new quantity will be $P - rP$, which can be factored as $\left(\answer{1-r}\right)P$.  In other words, when the decay rate is $r$, the decay factor is $\answer{1-r}$.  

\end{problem}
\end{problem}
\end{problem}
\end{problem}


\begin{problem}
Back to Metroville, where the population was $380,\!000$ in January $2024$, and it has been growing at $3\%$ per year.  

Write a recursive formula for this situation:  

\begin{prompt}
$f(n+1) =$ \wordChoice{\choice{$380,\!000$}\choice[correct]{$f(n)$}\choice{$f(n-1)$}\choice{$3\%$}}
\wordChoice{\choice{$+$}\choice{$-$}\choice[correct]{$\times$}\choice{$\div$}\choice{$\wedge$}}$\answer{1.03}$.
\end{prompt}
\begin{problem}
Using January 2024 as ``year 0,'' so that $f(0)=380000$,  write an explicit formula (without commas) to model this situation: 
\begin{prompt}
$f(t) = \answer{380000\cdot 1.03^t}$.
\end{prompt}

\begin{problem}
Use your explicit formula to compute the following values when the input value makes sense.  If the input value does not make sense in the domain of the value, enter `NA' without quotes.
\begin{enumerate}
\item $f(-104) = \answer{NA}$
\item $f(-3) = \answer[tolerance=100]{380000\cdot 1.03^{-3}}$
\item $f(2.5) = \answer[tolerance=100]{380000\cdot 1.03^{2.5}}$
\item $f(4) = \answer[tolerance=100]{380000\cdot 1.03^4}$
\item $f(220) = \answer{NA}$
\end{enumerate}
\begin{problem}
Correct!  Some of these make sense in the context.  For example, $f(-3)$ would approximate the population in January $\answer{2021}$, $f(4)$ would approximate the population in January $\answer{2028}$, and $f(2.5)$ would approximate the population in $\answer[format=string]{July}$ (month) $\answer{2026}$ (year).  

But $f(-104)$ would mean the population in $\answer{1920}$, before major world-changing events like World War II and the Great Depression.  Similarly, $f(220)$ would mean the population in $\answer{2444}$, and there are likely to be several world-changing events before then.  

\end{problem}

\end{problem}
\end{problem}

\end{problem}


\begin{problem}
Back to Kayla's new car, which she purchased for $\$16,\!000$ in January $2024$, and is estimated to depreciate $8\%$ per year.  

Write a recursive formula for this situation:  

\begin{prompt}
$f(n+1) =$ \wordChoice{\choice{$16,\!000$}\choice[correct]{$f(n)$}\choice{$f(n-1)$}\choice{$8\%$}}
\wordChoice{\choice{$+$}\choice{$-$}\choice[correct]{$\times$}\choice{$\div$}\choice{$\wedge$}}$\answer{0.92}$.
\end{prompt}
\begin{problem}
Using January 2024 as ``year 0,'' so that $f(0)=16000$,  write an explicit formula (without commas) to model this situation: 
\begin{prompt}
$f(t) = \answer{16000\cdot 0.92^t}$.
\end{prompt}

\begin{problem}
Use your explicit formula to compute the following values when the input value makes sense.  If the input value does not make sense in the domain of the value, enter `NA' without quotes.
\begin{enumerate}
\item $f(-3) = \answer{NA}$
\item $f(1.75) = \answer[tolerance=10]{16000\cdot 0.92^{1.75}}$
\item $f(6) = \answer[tolerance=10]{16000\cdot 0.92^6}$
\item $f(30) = \answer{NA}$
\end{enumerate}
\begin{problem}
Correct!  Some of these make sense in the context.  For example, $f(6)$ would approximate the car's value in January $\answer{2030}$, and $f(1.75)$ would approximate the car's value in $\answer[format=string]{October}$ (month) $\answer{2025}$ (year).  

But $f(-3)$ would mean the car's value in $\answer{2021}$, before it was built.  Also, $f(30)$ would mean the car's value in $\answer{2054}$, when it is unlikely to be still road-worthy.  If it is road-worthy, it will be an antique.    

\end{problem}

\end{problem}
\end{problem}
\end{problem}


\end{document}



