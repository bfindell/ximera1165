
\documentclass[nooutcomes]{ximera}
%\documentclass[space,handout,nooutcomes]{ximera}

% For preamble materials

\usepackage{pgf,tikz}
\usepackage{mathrsfs}
\usetikzlibrary{arrows}
\usepackage{framed}
\usepackage{amsmath}
%\pgfplotsset{compat=1.16}

\graphicspath{
  {./}
  {algorithms/}
  {../algorithms/}
}

\pdfOnly{\renewenvironment{image}[1][]{\begin{center}}{\end{center}}}

%%% This set of code is all of our user defined commands
\newcommand{\bysame}{\mbox{\rule{3em}{.4pt}}\,}
\newcommand{\N}{\mathbb N}
\newcommand{\C}{\mathbb C}
\newcommand{\W}{\mathbb W}
\newcommand{\Z}{\mathbb Z}
\newcommand{\Q}{\mathbb Q}
\newcommand{\R}{\mathbb R}
\newcommand{\A}{\mathbb A}
\newcommand{\D}{\mathcal D}
\newcommand{\F}{\mathcal F}
\newcommand{\ph}{\varphi}
\newcommand{\ep}{\varepsilon}
\newcommand{\aph}{\alpha}
\newcommand{\QM}{\begin{center}{\huge\textbf{?}}\end{center}}

\renewcommand{\le}{\leqslant}
\renewcommand{\ge}{\geqslant}
\renewcommand{\a}{\wedge}
\renewcommand{\v}{\vee}
\renewcommand{\l}{\ell}
\newcommand{\mat}{\mathsf}
\renewcommand{\vec}{\mathbf}
\renewcommand{\subset}{\subseteq}
\renewcommand{\supset}{\supseteq}
\renewcommand{\emptyset}{\varnothing}
\newcommand{\xto}{\xrightarrow}
\renewcommand{\qedsymbol}{$\blacksquare$}
\newcommand{\bibname}{References and Further Reading}
\renewcommand{\bar}{\protect\overline}
\renewcommand{\hat}{\protect\widehat}
\renewcommand{\tilde}{\widetilde}
\newcommand{\tri}{\triangle}
\newcommand{\minipad}{\vspace{1ex}}
\newcommand{\leftexp}[2]{{\vphantom{#2}}^{#1}{#2}}

%% More user defined commands
\renewcommand{\epsilon}{\varepsilon}
\renewcommand{\theta}{\vartheta} %% only for kmath
\renewcommand{\l}{\ell}
\renewcommand{\d}{\, d}
\newcommand{\ddx}{\frac{d}{dx}}
\newcommand{\dydx}{\frac{dy}{dx}}


\usepackage{bigstrut}


\newenvironment{sectionOutcomes}{}{}

\usepackage{array}
%\setlength{\extrarowheight}{-.2cm}   % Commented out by Findell to fix table headings.  Was this for typesetting division?  
\newdimen\digitwidth
\settowidth\digitwidth{9}
\def~{\hspace{\digitwidth}}
\def\divrule#1#2{
\noalign{\moveright#1\digitwidth
\vbox{\hrule width#2\digitwidth}}}


\title{Rational Numbers}
\author{Bart Snapp and Brad Findell}
\begin{document}
\begin{abstract}
More Problems about Rational Numbers.
\end{abstract}
\maketitle


%\begin{problem}
%Problem
%\begin{freeResponse}
%\begin{hint}
%Hint
%\end{hint}
%\end{freeResponse}
%\end{problem} 



\begin{problem}
Find yourself a sheet of paper. Now, suppose that this sheet of
  paper is actually $4/5$ of some imaginary larger sheet of
  paper. 
\begin{itemize}
\item Shade your sheet of paper so that $3/5$ of the larger
  (imaginary) sheet of paper is shaded in. Explain why your shading is
  correct.
\item Explain how this shows that 
\[
\frac{3/5}{4/5} = \frac{3}{4}.
\]
\end{itemize}
\end{problem}

\begin{problem}
Try to find the largest rational number smaller than $3/7$.
  Explain your solution or explain why this cannot be done.
\end{problem}

\begin{problem}
How many rational numbers are there between $3/4$ and $4/7$?
  Find $3$ of them. Explain your reasoning. 
\end{problem}

\begin{problem}
A youthful Bart loved to eat hamburgers. He ate $5/8$ pounds of
  hamburger meat a day. After testing revealed that his blood
  consisted mostly of cholesterol, Bart decided to alter his eating
  habits by cutting his hamburger consumption by $3/4$. How many
  pounds of hamburger a day did Bart eat on his new
  ``low-cholesterol'' diet?  Explain your reasoning.
\end{problem}

\begin{problem}
Courtney and Paolo are eating popcorn. Unfortunately, $1/3$rd 
  of the popcorn kernels are poisoned. If Courtney eats exactly $5/16$th 
  of the kernels and Paolo eats exactly $5/13$ths of the kernels, did at 
  least one of them eat a poisoned kernel?  Explain your reasoning.  Also, 
  at least how many kernels of popcorn are in the bowl? Again, explain 
  your reasoning.
\end{problem}

\begin{problem}
Best of clocks, how much of the day is past if there remains
  twice two-thirds of what is gone? Explain what this strange question
  is asking and answer the question being sure to explain your
  reasoning---note this is an old problem from the \textit{Greek
    Anthology} compiled by Metrodorus around the year 500.
\end{problem}

\begin{problem}
John spent a fifth of his life as a boy growing up, another
  one-sixth of his life in college, one-half of his life as a bookie,
  and has spent the last six years in prison. How old is John now?
  Explain your reasoning
\end{problem}

\begin{problem}
Diophantus was a boy for $1/6$th of his life, his beard grew
  after $1/12$ more, he married after $1/7$th more, and a son was born
  five years after his marriage. Alas! After attaining the measure of
  half his father's full life, chill fate took the child. Diophantus
  spent the last four years of his life consoling his grief through
  mathematics. How old was Diophantus when he died?  Explain your
  reasoning---note this is an old problem from the \textit{Greek
    Anthology} compiled by Metrodorus around the year 500.
\end{problem}

\begin{problem}
Wandering around my home town (perhaps trying to find my former
  self!), I suddenly realized that I had been in my job for
  one-quarter of my life. Perhaps the melancholia was getting the best
  of me, but I wondered: How long would it be until I had been in my
  job for one-third of my life? Explain your reasoning.
\end{problem}

\begin{problem}
In a certain adult condominium complex, $2/3$ of the men are
  married to $3/5$ of the women. Assuming that men are only married to
  women (and vice versa), and that married residents' spouses are also
  residents, what portion of the residents are married? 
\begin{enumerate}
\item Before any computations are done, use common sense to guess the
  solution to this problem.
\item Try to get a feel for this problem by choosing numbers for the
  unknowns and doing some calculations. What do these calculations say
  about your guess?
\item Use algebra to solve the problem.
\end{enumerate}
Explain your reasoning in each step above.

\end{problem}

\begin{problem}
Let $a$, $b$, $c$, and $d$ be positive integers such that 
\[
a<b<c<d
\]
Is it true that 
\[
\frac{a}{b}<\frac{c}{d}?
\]
Explain your reasoning.
\end{problem}

\begin{problem}
\label{P:CF1} Let $a$, $b$, $c$, and $d$ be positive consecutive
  integers such that
\[
a<b<c<d.
\]
Is it true that 
\[
\frac{a}{b}<\frac{c}{d}?
\]
Explain your reasoning.
\end{problem}

\begin{problem}
\label{P:CF2} Let $a$, $b$, $c$, and $d$ be positive consecutive
  integers such that
\[
a<b<c<d.
\]
Is it true that 
\[
\frac{a}{b}<\frac{b}{c}<\frac{c}{d}?
\]
Explain your reasoning.
\end{problem}

\begin{problem}
Can you generalize Problem \ref{P:CF1} and Problem \ref{P:CF2}
  above? Explain your reasoning.
\item Let $a$, $b$, $c$, and $d$ be positive integers such that 
\[
\frac{a}{b}<\frac{c}{d}.
\]
Is it true that 
\[
\frac{a}{a+b}<\frac{c}{c+d}?
\]
Explain your reasoning.
\end{problem}


\end{document}