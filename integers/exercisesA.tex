
\documentclass[nooutcomes]{ximera}
%\documentclass[space,handout,nooutcomes]{ximera}

% For preamble materials

\usepackage{pgf,tikz}
\usepackage{mathrsfs}
\usetikzlibrary{arrows}
\usepackage{framed}
\usepackage{amsmath}
%\pgfplotsset{compat=1.16}

\graphicspath{
  {./}
  {algorithms/}
  {../algorithms/}
}

\pdfOnly{\renewenvironment{image}[1][]{\begin{center}}{\end{center}}}

%%% This set of code is all of our user defined commands
\newcommand{\bysame}{\mbox{\rule{3em}{.4pt}}\,}
\newcommand{\N}{\mathbb N}
\newcommand{\C}{\mathbb C}
\newcommand{\W}{\mathbb W}
\newcommand{\Z}{\mathbb Z}
\newcommand{\Q}{\mathbb Q}
\newcommand{\R}{\mathbb R}
\newcommand{\A}{\mathbb A}
\newcommand{\D}{\mathcal D}
\newcommand{\F}{\mathcal F}
\newcommand{\ph}{\varphi}
\newcommand{\ep}{\varepsilon}
\newcommand{\aph}{\alpha}
\newcommand{\QM}{\begin{center}{\huge\textbf{?}}\end{center}}

\renewcommand{\le}{\leqslant}
\renewcommand{\ge}{\geqslant}
\renewcommand{\a}{\wedge}
\renewcommand{\v}{\vee}
\renewcommand{\l}{\ell}
\newcommand{\mat}{\mathsf}
\renewcommand{\vec}{\mathbf}
\renewcommand{\subset}{\subseteq}
\renewcommand{\supset}{\supseteq}
\renewcommand{\emptyset}{\varnothing}
\newcommand{\xto}{\xrightarrow}
\renewcommand{\qedsymbol}{$\blacksquare$}
\newcommand{\bibname}{References and Further Reading}
\renewcommand{\bar}{\protect\overline}
\renewcommand{\hat}{\protect\widehat}
\renewcommand{\tilde}{\widetilde}
\newcommand{\tri}{\triangle}
\newcommand{\minipad}{\vspace{1ex}}
\newcommand{\leftexp}[2]{{\vphantom{#2}}^{#1}{#2}}

%% More user defined commands
\renewcommand{\epsilon}{\varepsilon}
\renewcommand{\theta}{\vartheta} %% only for kmath
\renewcommand{\l}{\ell}
\renewcommand{\d}{\, d}
\newcommand{\ddx}{\frac{d}{dx}}
\newcommand{\dydx}{\frac{dy}{dx}}


\usepackage{bigstrut}


\newenvironment{sectionOutcomes}{}{}

\usepackage{array}
%\setlength{\extrarowheight}{-.2cm}   % Commented out by Findell to fix table headings.  Was this for typesetting division?  
\newdimen\digitwidth
\settowidth\digitwidth{9}
\def~{\hspace{\digitwidth}}
\def\divrule#1#2{
\noalign{\moveright#1\digitwidth
\vbox{\hrule width#2\digitwidth}}}


\title{Integers}
\author{Bart Snapp and Brad Findell}
\begin{document}
\begin{abstract}
Problems about integers.
\end{abstract}
\maketitle


%\begin{problem}
%Problem
%\begin{freeResponse}
%\begin{hint}
%Hint
%\end{hint}
%\end{freeResponse}
%\end{problem} 


%\begin{problem}
%Describe the set of integers. Give some relevant and revealing
%  examples/nonexamples.
%\begin{freeResponse}
%\begin{hint}
%The integers are the counting numbers, 0, and the opposites of the counting numbers.  
%\[\{ \dots, -3, -2, -1, 0, 1, 2, 3, \dots\}\]
%\end{hint}
%\end{freeResponse}
%\end{problem}

%\begin{problem}
%Explain how to model integer addition with pictures or
%  items. What relevant properties should your model show?
%\begin{freeResponse}
%\begin{hint}
%Hint
%\end{hint}
%\end{freeResponse}
%\end{problem}
%
%\begin{problem}
%Explain how to model integer multiplication with pictures or
%  items. What relevant properties should your model show?
%\end{problem}
%
%\begin{problem}
%Explain what it means for one integer to \textit{divide} another
%  integer. Give some relevant and revealing examples/nonexamples.
%\begin{freeResponse}
%\begin{hint}
%Hint
%\end{hint}
%\end{freeResponse}
%\end{problem}

\begin{problem}
Use the definition of \textit{divides} to decide whether the
  following statements are true or false. In each case, an explanation must 
be given justifying your claim.
\begin{enumerate}
\item $5|30$  \wordChoice{\choice[correct]{True}\choice{False}}
  \begin{hint}$30=5\cdot6$.\end{hint}
\item $7|41$  \wordChoice{\choice{True}\choice[correct]{False}}
  \begin{hint}There is no integer solution to $41=7k$.  But $41=7\cdot5+6$. \end{hint}
\item $0|3$  \wordChoice{\choice{True}\choice[correct]{False}}
  \begin{hint}There is no integer solution to $3=0k$.\end{hint}
\item $3|0$ \wordChoice{\choice[correct]{True}\choice{False}}
  \begin{hint}The solution to $0=3k$ is $k=0$.\end{hint}
\item $6|(2^2\cdot 3^4\cdot 5 \cdot 7)$. \wordChoice{\choice[correct]{True}\choice{False}}
  \begin{hint} $6=2\cdot3$, and $2\cdot3$ appears in factorization of the second number.\end{hint}
\item $1000|(2^7\cdot 3^9\cdot 5^{11}\cdot 17^8)$ \wordChoice{\choice[correct]{True}\choice{False}}
  \begin{hint} $1000 = 2^3\cdot5^3$, and these primes appear enough times in factorization of the second number.\end{hint}
\item $6000|(2^{21}\cdot 3^{17}\cdot 5^{89}\cdot 29^{20})$. \wordChoice{\choice[correct]{True}\choice{False}}
  \begin{hint} $6000 = 2^4\cdot3\cdot5^3$, and these primes appear enough times in factorization of the second number.\end{hint}
\end{enumerate}
\end{problem}

%\begin{problem}
%\textit{Incognito's Hall of Shoes} is a shoe store that just
%  opened in Myrtle Beach, South Carolina. At the moment, they have 100
%  pairs of shoes in stock. At their grand opening 100 customers showed
%  up. The first customer tried on every pair of shoes, the second
%  customer tried on every 2nd pair, the third customer tried on every
%  3rd pair, and so on until the 100th customer, who only tried on the
%  last pair of shoes.
%\begin{enumerate}
%\item Which shoes were tried on by only 1 customer?
%\item Which shoes were tried on by exactly 2 customers?
%\item Which shoes were tried on by exactly 3 customers?
%\item Which shoes were tried on by the most number of customers?
%\end{enumerate}
%Explain your reasoning.
%\end{problem}


\begin{javascript}
function isPrime(num) {
  for(var i = 2; i < num; i++)
    if(num % i === 0) return false;
  return num > 1;
}

function isPrimeFactorization(x,y) {
  var terms = x.split('*').map( function(t) { return parseInt(t) } );
  return terms.every( isPrime ) &&
    (terms.reduce( function(a,c) { return a*c; }, 1 )) == parseInt(y);
}
\end{javascript}

%Then $\answer[format=string,validator=isPrimeFactorization]{33}$ would
%accept 3*11 and 11*3.  This doesn't work with powers or signs.


\begin{problem}
Factor the following integers.  Enter the primes in increasing order, use * for multiplication, and do not use exponents.  If the number is prime, enter the number itself.
\begin{enumerate}
\item $15\qquad$ $3*5$
\item $12\qquad$ $2*2*3$
\item $111$     $\answer[format=string]{3*37}$
\item $1234$    $\answer[format=string]{2*617}$
\item $2345$    $\answer[format=string]{5*7*67}$
\item $4567$    $\answer[format=string]{4567}$
\item $111111$  $\answer[format=string]{3*7*11*13*37}$
\end{enumerate}
%In each case, how large a prime must you check before you can be sure
%of your answers? Explain your reasoning.
\end{problem}

%\begin{problem}
%Which of the following numbers are prime?  Explain how could deduce whether the numbers are prime in
%  as few calculations as possible:
%\[
%29 \qquad 53 \qquad 101 \qquad 359 \qquad 779 \qquad 839 \qquad 841
%\]
%In each case, describe precisely which computations are needed and
%why those are the only computations needed.
%\end{problem}
%
%\begin{problem}
%Suppose you were only allowed to perform at most $7$
%  computations to see if a number is prime. How large a number could
%  you check?  Explain your reasoning.
%\end{problem}
%
%\begin{problem}
%Find examples of integers $a$, $b$, and $c$ such that $a \mid
%  bc$ but $a\nmid b$ and $a\nmid c$. Explain your reasoning.
%\end{problem}
%
%\begin{problem}
%Can you find at least $5$ composite integers in a row? What
%  about at least $6$ composite integers? Can you find $7$?
%  What about $n$?  Explain your reasoning. Hint: Consider something
%  like $5! = 5\cdot 4 \cdot 3 \cdot 2 \cdot 1$.
%\end{problem}

\begin{problem}
% Use the definition of the \textit{greatest common divisor} to
%  find the GCD of each of the pairs below. 
%  In each case, a detailed argument and explanation must 
%  be given justifying your claim.
Find the greatest common divisors below:  
\begin{enumerate}
\item $\gcd(462,1463) = \answer{77}$  
  \begin{hint} $462 = 2\cdot 3\cdot 7\cdot 11$, and  $1463 = 7\cdot 11\cdot 19$.\end{hint}
\item $\gcd(541,4669) = \answer{1}$. 
  \begin{hint}$541$ is prime.  And $4669 = 7\cdot 23\cdot 29$.\end{hint}
\item $\gcd(10000,2^5\cdot 3^{19}\cdot 5^7\cdot 11^{13}) = \answer{10000}$
  \begin{hint}$10000 = 2^5\cdot5^5$.\end{hint}
\item $\gcd(11111,2^{14}\cdot 7^{21}\cdot 41^{5}\cdot 101) = \answer{41}$
  \begin{hint}$11111 = 41\cdot 271$.\end{hint}
\item $\gcd(437^5,8993^3) = \answer{23^5}$
  \begin{hint}$437 = 19\cdot 23$, and $8993=17\cdot 23^2$.\end{hint}
\end{enumerate}
\end{problem}

%\begin{problem}
%Lisa wants to make a new quilt out of $2$ of her favorite
%  sheets. To do this, she is going to cut each sheet into as large
%  squares as possible while using the entire sheet and using whole
%  inch measurements. 
%\begin{enumerate}
%\item If the first sheet is $72$ inches by $60$ inches what size
%  squares should she cut? 
%\item If the second sheet is $80$ inches by $75$ inches, what size
%  squares should she cut? 
%\item How she might sew these squares together? 
%\end{enumerate}
%Explain your reasoning.
%\end{problem}
%
%\begin{problem}
%Deena and Doug like to feed birds. They want to put 16 cups of
%  millet seed and 24 cups of sunflower seeds in their feeder.
%\begin{enumerate}
%\item How many total scoops of seed (millet or sunflower) are required
%  if their scoop holds 1 cup of seed?
%\item How many total scoops of seed (millet or sunflower) are required
%  if their scoop holds 2 cups of seed?
%\item How large should the scoop be if we want to minimize the total
%  number of scoops?
%\end{enumerate}
%Explain your reasoning.
%\end{problem}
%
%\begin{problem}
%Consider the expression:
%\[
%d\,\begin{tabular}[b]{@{}r@{} r}
%$q$ &\, R\,$r$\\ \cline{1-1}
%\big)\begin{tabular}[t]{@{}l@{}}
%$n$ 
%\end{tabular}
%\end{tabular}
%\qquad\text{where}\qquad
%\begin{tabular}{l}
%$d$ is the divisor \\
%$n$ is the dividend \\
%$q$ is the quotient \\
%$r$ is the remainder
%\end{tabular}
%\]
%\begin{enumerate}
%\item Give $3$ relevant and revealing examples of long division with
%  remainders.
%\item Given positive integers $d$, $n$, $q$, and $r$ how do you know
%  if they leave us with a correct expression above?
%\item Given positive integers $d$ and $n$, how many different sets of
%  $q$ and $r$ can you find that will leave us with a correct
%  expression above?
%\item Give $3$ relevant and revealing examples of long division with
%  remainders where some of $d$, $n$, $q$, and $r$ are negative.
%\item Still allowing some of $d$, $n$, $q$, and $r$ to be negative,
%  how do we know if they leave us with a correct expression above?
%\end{enumerate}
%\end{problem}
%
%\begin{problem}
%State the \textit{Division Theorem} for integers. Give some
%  relevant and revealing examples of this theorem in action.
%\item Explain what it means for an integer to \textit{not} divide
%  another integer. That is, explain symbolically what it should mean
%  to write:
%\[
%a \nmid b
%\]
%\end{problem}

\begin{problem}
Consider the following:
\begin{align*}
20 \div 8 &= 2 \text{ remainder }4, \\
28 \div 12 &= 2 \text{ remainder }4.
\end{align*}
Is it correct to say that $20 \div 8 = 28 \div 12$? \wordChoice{\choice{Yes}\choice[correct]{No}}

Explain your reasoning.
\begin{freeResponse}
\begin{hint}
The answer ``$2 \text{ remainder }4$'' is not a single number but rather a pair of numbers (a quotient and a remainder) that have different meanings.  In particular, the 2 is about different things: groups of 8 versus groups of 12.  Calling this pair of numbers ``equal'' is questionable.  
\end{hint}
\end{freeResponse}

%\begin{problem} Explain how if 
%\[
%n = dq_0 + r_0
%\]
%and $r_0> d$, you can find new $q$ and $r$ such that $n = dq+r$ and
%$0\le r< d$.
\end{problem}

\begin{problem}
Give a formula for the $n$th even number: $\answer{2n}$
% Show-off your formula with some examples.  
\end{problem}

\begin{problem}
Give a formula for the $n$th odd number: $\answer{2n-1}$. 
\end{problem}

\begin{problem}
Give a formula for the $n$th multiple of $3$:  $\answer{3n}$
\end{problem}

\begin{problem}
Give a formula for the $n$th multiple of $-7$. $\answer{-7n}$
\end{problem}

\begin{problem}
Give a formula for the $n$th number whose remainder when divided
  by $5$ is $1$. 
  
  If the first such number is 1, the formula is $\answer{5n-4}$. 

  If the first such number is 6, the formula is $\answer{5n+1}$. 
\end{problem}
%
%\begin{problem}
%Explain the rule
%\[
%\text{even} + \text{even} = \text{even}
%\]
%in two different ways. First give an explanation based on
%pictures. Second give an explanation based on algebra. Your
%explanations must be general, not based on specific examples.
%\end{problem}
%
%\begin{problem}
%Explain the rule
%\[
%\text{odd} + \text{even} = \text{odd}
%\]
%in two different ways. First give an explanation based on
%pictures. Second give an explanation based on algebra.  Your
%explanations must be general, not based on specific examples.
%\end{problem}
%
%\begin{problem}
%Explain the rule
%\[
%\text{odd} + \text{odd} = \text{even}
%\]
%in two different ways. First give an explanation based on
%pictures. Second give an explanation based on algebra. Your
%explanations must be general, not based on specific examples.
%\end{problem}
%
%\begin{problem}
%Explain the rule
%\[
%\text{even} \cdot \text{even} = \text{even}
%\]
%in two different ways. First give an explanation based on
%pictures. Second give an explanation based on algebra. Your
%explanations must be general, not based on specific examples.
%\end{problem}
%
%\begin{problem}
%Explain the rule
%\[
%\text{odd} \cdot \text{odd} = \text{odd}
%\]
%in two different ways. First give an explanation based on
%pictures. Second give an explanation based on algebra. Your
%explanations must be general, not based on specific examples.
%\end{problem}
%
%\begin{problem}
%Explain the rule
%\[
%\text{odd} \cdot \text{even} = \text{even}
%\]
%in two different ways. First give an explanation based on
%pictures. Second give an explanation based on algebra. Your
%explanations must be general, not based on specific examples.
%\end{problem}
%
%\begin{problem}
%Let $a\ge b$ be positive integers with $\gcd(a,b) =1$. Compute
%  $\gcd(a +b, a-b)$. Explain your reasoning. Hints: 
%\begin{enumerate}
%\item Make a chart.
%\item If $g|x$ and $g|y$ explain why $g|(x+y)$.
%\end{enumerate}
%\end{problem}
%
%\begin{problem}
%Make a chart listing all pairs of positive integers whose
%  product is $18$. Do the same for $221$, $462$, and $924$. Use this
%  experience to help you explain why when factoring a number $n$, you
%  only need to check factors less than or equal to $\sqrt{n}$.
%\end{problem}
%
%\begin{problem}
%\label{P:NNP}Matt is a member of the Ohio State University
%  Marching Band. Being rather capable, Matt can take $x$ steps of size
%  $y$ inches for all integer values of $x$ and $y$.  If $x$ is
%  positive it means \textit{face North and take $x$ steps.} If $x$ is
%  negative it means \textit{face South and take $|x|$ steps.} If $y$
%  is positive it means your step is a \textit{forward step of $y$
%    inches.} If $y$ is negative it means your step \textit{is a
%    backward step of $|y|$ inches.}
%    
%% \fixnote{We need additional models, e.g., checks and bills, red and black chips. 
%%    Some of these are incorporated into Activities \ref{A:integerAddition} and \ref{A:integerMultiplication}}
%\begin{enumerate}
%\item Discuss what the expressions $x \cdot y$ means in this
%  context. In particular, what happens if $x = 1$? What if $y=1$?
%\item Using the context above, write and solve a word problem that
%  demonstrates the rule:
%\[
%\text{negative}\cdot \text{positive} = \text{negative}
%\]
%Clearly explain how your problem shows this.
%\item Using the context above, write and solve a word problem that
%  demonstrates the rule:
%\[
%\text{negative}\cdot \text{negative} = \text{positive}
%\]
%Clearly explain how your problem shows this.
%\end{enumerate}
%\end{problem}
%
%\begin{problem}
%Stewie decided to count the pennies he had in his piggy bank. He
%  decided it would be quicker to count by fives. However, he ended
%  with two uncounted pennies. So he tried counting by twos but ended
%  up with one uncounted penny. Next he counted by threes and then by
%  fours, each time there was one uncounted penny. Though he knew he
%  had less than a dollars worth of pennies, and more than 50 cents, he
%  still didn't have an exact count. Can you help Stewie out? Explain
%  your reasoning.
%\end{problem}



\end{document}
