
\documentclass[nooutcomes]{ximera}
%\documentclass[space,handout,nooutcomes]{ximera}

% For preamble materials

\graphicspath{
  {./}
  {algorithms/}
  {../algorithms/}
}


%%% This set of code is all of our user defined commands
\newcommand{\bysame}{\mbox{\rule{3em}{.4pt}}\,}
\newcommand{\N}{\mathbb N}
\newcommand{\C}{\mathbb C}
\newcommand{\W}{\mathbb W}
\newcommand{\Z}{\mathbb Z}
\newcommand{\Q}{\mathbb Q}
\newcommand{\R}{\mathbb R}
\newcommand{\A}{\mathbb A}
\newcommand{\D}{\mathcal D}
\newcommand{\F}{\mathcal F}
\newcommand{\ph}{\varphi}
\newcommand{\ep}{\varepsilon}
\newcommand{\aph}{\alpha}
\newcommand{\QM}{\begin{center}{\huge\textbf{?}}\end{center}}

\renewcommand{\le}{\leqslant}
\renewcommand{\ge}{\geqslant}
\renewcommand{\a}{\wedge}
\renewcommand{\v}{\vee}
\renewcommand{\l}{\ell}
\newcommand{\mat}{\mathsf}
\renewcommand{\vec}{\mathbf}
\renewcommand{\subset}{\subseteq}
\renewcommand{\supset}{\supseteq}
\renewcommand{\emptyset}{\varnothing}
\newcommand{\xto}{\xrightarrow}
\renewcommand{\qedsymbol}{$\blacksquare$}
\newcommand{\bibname}{References and Further Reading}
\renewcommand{\bar}{\protect\overline}
\renewcommand{\hat}{\protect\widehat}
\renewcommand{\tilde}{\widetilde}
\newcommand{\tri}{\triangle}
\newcommand{\minipad}{\vspace{1ex}}
\newcommand{\leftexp}[2]{{\vphantom{#2}}^{#1}{#2}}

%% More user defined commands
\renewcommand{\epsilon}{\varepsilon}
\renewcommand{\theta}{\vartheta} %% only for kmath
\renewcommand{\l}{\ell}
\renewcommand{\d}{\, d}
\newcommand{\ddx}{\frac{d}{dx}}
\newcommand{\dydx}{\frac{dy}{dx}}


\usepackage{bigstrut}


\newenvironment{sectionOutcomes}{}{}

\usepackage{array}
%\setlength{\extrarowheight}{-.2cm}   % Commented out by Findell to fix table headings.  Was this for typesetting division?  
\newdimen\digitwidth
\settowidth\digitwidth{9}
\def~{\hspace{\digitwidth}}
\def\divrule#1#2{
\noalign{\moveright#1\digitwidth
\vbox{\hrule width#2\digitwidth}}}


\title{Counting}
\author{Bart Snapp and Brad Findell and Jenny Sheldon}
\begin{document}
\begin{abstract}
Problems about systematic ordering.
\end{abstract}
\maketitle


Easy counting problems are easy.  But in challenging counting problems, it becomes very important to use a systematic method for listing all of the possibilities.  Without a systematic approach, it is to easy to miss some possibilities or to    
list other possibilities more than once (or both). 

Here are some strategies that are very useful in solving such problems:  
\begin{itemize}
\item Fix one choice, and systematically change the other parts through all possibilities.  Change the first choice to the next option, and repeat.  
\item List the possibilities in dictionary or numerical order, either forward or backward. 
\item Organize the possibilities into ``types,'' and then be systematic within each type.  
\item Intentionally overcount, and then adjust systematically.  
\end{itemize}
These strategies can be combined.  

Some examples are given below. 

\newpage 
\begin{problem}
List all of the factors of $5^3\cdot7^2\cdot11$. 

\begin{solution}
\[
\begin{array}{c}
5^0\cdot7^0\cdot11^0 \\
5^0\cdot7^0\cdot11^1 \\
5^0\cdot7^1\cdot11^0 \\
\answer{5^0\cdot7^1\cdot11^1} \\
\answer{5^0\cdot7^2\cdot11^0} \\
\answer{5^0\cdot7^2\cdot11^1} \\ \\
5^1\cdot7^0\cdot11^0 \\
\answer{5^1\cdot7^0\cdot11^1} \\
\answer{5^1\cdot7^1\cdot11^0} \\
\answer{5^1\cdot7^1\cdot11^1} \\
\answer{5^1\cdot7^2\cdot11^0} \\
\answer{5^1\cdot7^2\cdot11^1} \\ \\
5^2\cdot7^0\cdot11^0 \\
\answer{5^2\cdot7^0\cdot11^1} \\
\answer{5^2\cdot7^1\cdot11^0} \\
\answer{5^2\cdot7^1\cdot11^1} \\
\answer{5^2\cdot7^2\cdot11^0} \\
\answer{5^2\cdot7^2\cdot11^1} \\ \\
\answer{5^3\cdot7^0\cdot11^0} \\
\answer{5^3\cdot7^0\cdot11^1} \\
\answer{5^3\cdot7^1\cdot11^0} \\
\answer{5^3\cdot7^1\cdot11^1} \\
\answer{5^3\cdot7^2\cdot11^0} \\
5^3\cdot7^2\cdot11^1
\end{array}
\]
\end{solution}
\end{problem}

\newpage
\begin{problem}
List all right-rectangular prisms of volume 24 and whole-number sides.  Assume the three dimensions are distinguishable. 
\begin{solution}
\[
\begin{array}{|c|c|c|}
\hline
\text{length} & \text{width} & \text{height} \\ 
\hline
1 & 1 & 24 \\
1 & 2 & 12 \\
1 & 3 & 8 \\
1 & 4 & 6 \\
1 & 6 & 4 \\
1 & 8 & 3 \\
1 & 12 & 2 \\
1 & 24 & 1 \\
2 & 1 & 12 \\
2 & 2 & 6 \\
2 & 3 & 4 \\
2 & 4 & 3 \\
2 & 6 & 2 \\
2 & 12 & 1 \\
3 & 1 & 8 \\
3 & 2 & 4 \\
3 & 4 & 2 \\
3 & 8 & 1 \\
4 & 1 & 6 \\
4 & 2 & 3 \\
4 & 3 & 2 \\
4 & 6 & 1 \\
6 & 1 & 4 \\
6 & 2 & 2 \\
6 & 4 & 1 \\
8 & 1 & 3 \\
8 & 3 & 1 \\
12 & 1 & 2 \\
12 & 2 & 1 \\
24 & 1 & 1 \\
\hline
\end{array}
\]
\end{solution}
\end{problem}

\newpage
\begin{problem}
List all right-rectangular prisms of volume 24 and whole-number sides.  This time, the three dimensions are not distinguishable. 

\begin{solution}
To prevent overcounting, we require that 
\[
\text{length} \le \text{width} \le \text{height}. 
\]
\[
\begin{array}{|c|c|c|}
\hline
\text{length} & \text{width} & \text{height} \\ 
\hline
1 & 1 & 24 \\
\answer{1} & \answer{2} & \answer{12} \\
\answer{1} & \answer{3} & \answer{8} \\
\answer{1} & \answer{4} & \answer{6} \\
\answer{2} & \answer{2} & \answer{6} \\
\answer{2} & \answer{3} & \answer{4} \\
\hline
\end{array}
\]
\end{solution}
\end{problem}


\newpage
\begin{problem}
In the red/green light context, list all combinations of 2 red lights from among 5 lights.    
\begin{solution}
List in dictionary order: 
\[
\begin{array}{c}
GGGRR \\
\answer[format=string]{GGRGR} \\
\answer[format=string]{GGRRG} \\
\answer[format=string]{GRGGR} \\
\answer[format=string]{GRGRG} \\
\answer[format=string]{GRRGG} \\
\answer[format=string]{RGGGR} \\
\answer[format=string]{RGGRG} \\
\answer[format=string]{RGRGG} \\
\answer[format=string]{RRGGG}
\end{array}
\]
\end{solution}
\end{problem}


\end{document}



