
\documentclass[nooutcomes]{ximera}
%\documentclass[space,handout,nooutcomes]{ximera}

% For preamble materials

\usepackage{pgf,tikz}
\usepackage{mathrsfs}
\usetikzlibrary{arrows}
\usepackage{framed}
\usepackage{amsmath}
%\pgfplotsset{compat=1.16}

\graphicspath{
  {./}
  {algorithms/}
  {../algorithms/}
}

\pdfOnly{\renewenvironment{image}[1][]{\begin{center}}{\end{center}}}

%%% This set of code is all of our user defined commands
\newcommand{\bysame}{\mbox{\rule{3em}{.4pt}}\,}
\newcommand{\N}{\mathbb N}
\newcommand{\C}{\mathbb C}
\newcommand{\W}{\mathbb W}
\newcommand{\Z}{\mathbb Z}
\newcommand{\Q}{\mathbb Q}
\newcommand{\R}{\mathbb R}
\newcommand{\A}{\mathbb A}
\newcommand{\D}{\mathcal D}
\newcommand{\F}{\mathcal F}
\newcommand{\ph}{\varphi}
\newcommand{\ep}{\varepsilon}
\newcommand{\aph}{\alpha}
\newcommand{\QM}{\begin{center}{\huge\textbf{?}}\end{center}}

\renewcommand{\le}{\leqslant}
\renewcommand{\ge}{\geqslant}
\renewcommand{\a}{\wedge}
\renewcommand{\v}{\vee}
\renewcommand{\l}{\ell}
\newcommand{\mat}{\mathsf}
\renewcommand{\vec}{\mathbf}
\renewcommand{\subset}{\subseteq}
\renewcommand{\supset}{\supseteq}
\renewcommand{\emptyset}{\varnothing}
\newcommand{\xto}{\xrightarrow}
\renewcommand{\qedsymbol}{$\blacksquare$}
\newcommand{\bibname}{References and Further Reading}
\renewcommand{\bar}{\protect\overline}
\renewcommand{\hat}{\protect\widehat}
\renewcommand{\tilde}{\widetilde}
\newcommand{\tri}{\triangle}
\newcommand{\minipad}{\vspace{1ex}}
\newcommand{\leftexp}[2]{{\vphantom{#2}}^{#1}{#2}}

%% More user defined commands
\renewcommand{\epsilon}{\varepsilon}
\renewcommand{\theta}{\vartheta} %% only for kmath
\renewcommand{\l}{\ell}
\renewcommand{\d}{\, d}
\newcommand{\ddx}{\frac{d}{dx}}
\newcommand{\dydx}{\frac{dy}{dx}}


\usepackage{bigstrut}


\newenvironment{sectionOutcomes}{}{}

\usepackage{array}
%\setlength{\extrarowheight}{-.2cm}   % Commented out by Findell to fix table headings.  Was this for typesetting division?  
\newdimen\digitwidth
\settowidth\digitwidth{9}
\def~{\hspace{\digitwidth}}
\def\divrule#1#2{
\noalign{\moveright#1\digitwidth
\vbox{\hrule width#2\digitwidth}}}


\title{Home Base}
\author{Bart Snapp and Brad Findell}
\begin{document}
\begin{abstract}
Problems about numbers in various bases. 
\end{abstract}
\maketitle

%For practice counting in bases other than ten, go to http://tube.geogebra.org/m/1529377.
%https://www.geogebra.org/m/Cz5myDUU
%
%For practice converting between bases, go to http://tube.geogebra.org/m/1527705.
%https://www.geogebra.org/m/ZEdZjfxm
%

If you haven't already practiced, take an opportunity now.  

\begin{center}
\geogebra{1529377}{995}{370}
\end{center}

\begin{center}
\geogebra{1527705}{780}{240}
\end{center}



\begin{problem}Explain why the following ``joke'' is ``funny:'' \textit{There
  are $10$ types of people in the world. Those who understand base two
  and those who don't.}
\begin{freeResponse}
\begin{hint}
In base two, 10 is actually two.  So people who do not understand base two will not get the joke.  
\end{hint}
\end{freeResponse}
\end{problem} 

\begin{problem}You meet some Tripod aliens, they tally by threes. Thankfully
  for everyone involved, they use the symbols $0$, $1$, and $2$. 
\begin{enumerate}
\item Can you explain how a Tripod would count from $11$ to $201$? Be
  sure to carefully explain what number comes after $22$.
  
12, 20, 21  $\answer[format=string]{12, 20, 21}$
(22, 100, 101)  $\answer{(22, 100, 101)}$

$\answer{102}, \answer{110}, \answer{111}, \answer{112}, \answer{120}, \answer{121}, \answer{122}, \answer{200}, \answer{201}$
  
\item What number comes immediately before $10$?  $\answer{2}$
\item Before $210$? $\answer{202}$
\item Before $20110$? $\answer{20102}$
  Explain your reasoning.
\end{enumerate}
\end{problem} 

\begin{problem}You meet some people who tally by sevens. They use the symbols
  $O$, $A$, $B$, $C$, $D$, $E$, and $F$. 
\begin{enumerate}
\item What do the individual symbols $O$, $A$, $B$, $C$, $D$, $E$, and
  $F$ mean?
\item Can you explain how they would count from $DD$ to $AOC$? Be sure
  to carefully explain what number comes after $FF$.
\item What number comes immediately before $AO$?  $ABO$? $EOFFA$?
  Explain your reasoning.
\end{enumerate}
\end{problem} 

\begin{problem}Now, suppose that you meet a hermit who tallies by
  thirteens. Explain how he might count. Give some relevant and
  revealing examples.
% The following problem used to begin $1/6$ of $30$ is $4$, but that allowed  
% proportional reasoning to give the correct answer.  The numbers have been changed.  
\end{problem} 


\end{document}