
\documentclass[nooutcomes]{ximera}
%\documentclass[space,handout,nooutcomes]{ximera}

% For preamble materials

\graphicspath{
  {./}
  {algorithms/}
  {../algorithms/}
}


%%% This set of code is all of our user defined commands
\newcommand{\bysame}{\mbox{\rule{3em}{.4pt}}\,}
\newcommand{\N}{\mathbb N}
\newcommand{\C}{\mathbb C}
\newcommand{\W}{\mathbb W}
\newcommand{\Z}{\mathbb Z}
\newcommand{\Q}{\mathbb Q}
\newcommand{\R}{\mathbb R}
\newcommand{\A}{\mathbb A}
\newcommand{\D}{\mathcal D}
\newcommand{\F}{\mathcal F}
\newcommand{\ph}{\varphi}
\newcommand{\ep}{\varepsilon}
\newcommand{\aph}{\alpha}
\newcommand{\QM}{\begin{center}{\huge\textbf{?}}\end{center}}

\renewcommand{\le}{\leqslant}
\renewcommand{\ge}{\geqslant}
\renewcommand{\a}{\wedge}
\renewcommand{\v}{\vee}
\renewcommand{\l}{\ell}
\newcommand{\mat}{\mathsf}
\renewcommand{\vec}{\mathbf}
\renewcommand{\subset}{\subseteq}
\renewcommand{\supset}{\supseteq}
\renewcommand{\emptyset}{\varnothing}
\newcommand{\xto}{\xrightarrow}
\renewcommand{\qedsymbol}{$\blacksquare$}
\newcommand{\bibname}{References and Further Reading}
\renewcommand{\bar}{\protect\overline}
\renewcommand{\hat}{\protect\widehat}
\renewcommand{\tilde}{\widetilde}
\newcommand{\tri}{\triangle}
\newcommand{\minipad}{\vspace{1ex}}
\newcommand{\leftexp}[2]{{\vphantom{#2}}^{#1}{#2}}

%% More user defined commands
\renewcommand{\epsilon}{\varepsilon}
\renewcommand{\theta}{\vartheta} %% only for kmath
\renewcommand{\l}{\ell}
\renewcommand{\d}{\, d}
\newcommand{\ddx}{\frac{d}{dx}}
\newcommand{\dydx}{\frac{dy}{dx}}


\usepackage{bigstrut}


\newenvironment{sectionOutcomes}{}{}

\usepackage{array}
%\setlength{\extrarowheight}{-.2cm}   % Commented out by Findell to fix table headings.  Was this for typesetting division?  
\newdimen\digitwidth
\settowidth\digitwidth{9}
\def~{\hspace{\digitwidth}}
\def\divrule#1#2{
\noalign{\moveright#1\digitwidth
\vbox{\hrule width#2\digitwidth}}}


\title{Home Base, Part A}
\author{Bart Snapp and Brad Findell}
\begin{document}
\begin{abstract}
Beginning problems about numbers in various bases. 
\end{abstract}
\maketitle

%For practice counting in bases other than ten, go to http://tube.geogebra.org/m/1529377.
%https://www.geogebra.org/m/Cz5myDUU
%
%For practice converting between bases, go to http://tube.geogebra.org/m/1527705.
%https://www.geogebra.org/m/ZEdZjfxm
%

If you haven't already, take an opportunity now to practice \link[counting]{https://www.geogebra.org/m/Cz5myDUU} in other bases and \link[converting]{https://www.geogebra.org/m/ZEdZjfxm} from base ten to other bases and vice versa.    

%\begin{center}
%\geogebra{Cz5myDUU}{900}{320}
%\end{center}
%
%\begin{center}
%\geogebra{ZEdZjfxm}{780}{240}
%\end{center}
%


\begin{problem}
Note:  The ``free response'' answers are not checked for accuracy.  To optimize your learning, we recommend you submit your own answer before revealing the hint.  

Complete the following sentence: 

To \wordChoice{\choice[correct]{optimize}\choice{minimize}} my learning, I plan to $\answer[format=string]{submit}$ my own answer \wordChoice{\choice[correct]{before}\choice{after}} revealing the $\answer[format=string]{hint}$.  
\end{problem}

\begin{problem}Explain why the following ``joke'' is ``funny:'' \textit{There
  are $10$ types of people in the world. Those who understand base two
  and those who don't.}
\begin{freeResponse}
\begin{hint}
In base two, 10 is actually two.  So people who do not understand base two will not get the joke.  
\end{hint}
\end{freeResponse}
\end{problem} 

\begin{problem}You meet some Tripod aliens, they tally by threes. Thankfully
  for everyone involved, they use the symbols $0$, $1$, and $2$. 
\begin{enumerate}
%\item Can you explain how a Tripod would count from $11$ to $201$? Be
%  sure to carefully explain what number comes after $22$.
  \item Demonstrate how a Tripod would count, beginning at $11$.  
  
$11, \answer{12}, \answer{20}, \answer{21}, \answer{22}, \answer{100}, \answer{101}, 
\answer{102}, \\
\answer{110}, \answer{111}, \answer{112}, \answer{120}, \answer{121}, \answer{122}, \answer{200}, \answer{201}$

\item What number comes immediately before $10$?  $\answer{2}$
\item Before $210$? $\answer{202}$
\item Before $20110$? $\answer{20102}$
  Explain your reasoning.
\end{enumerate}
\end{problem} 

%\begin{problem}You meet some people who tally by sevens. They use the symbols
%  $O$, $A$, $B$, $C$, $D$, $E$, and $F$, in that order. (Note: Although it is common to use the letters $A$ through $F$ for digits greater than ten, these people are doing something different.)
%\begin{enumerate}
%\item What do the individual symbols $O$, $A$, $B$, $C$, $D$, $E$, and
%  $F$ mean?  (Note for computer entry: $O$ and $0$ are different symbols.)
%  \begin{freeResponse}
%    \begin{hint}
%      0, 1, 2, 3, 4, 5, and 6, respectively.
%    \end{hint}
%  \end{freeResponse}
%\item Demonstrate how to count from $DD$ to $AOC$?
%
%$DD, \answer[format=string]{DE}, \answer[format=string]{DF}, \answer[format=string]{EO}, \answer[format=string]{EA}, \answer[format=string]{EB}, \\
%\answer[format=string]{EC}, \answer[format=string]{ED}, \answer[format=string]{EE}, 
%\answer[format=string]{EF}, \answer[format=string]{FO}, \\
%\answer[format=string]{FA}, \answer[format=string]{FB}, \answer[format=string]{FC}, \answer[format=string]{FD}, \answer[format=string]{FE}, \\
%\answer[format=string]{FF}, \answer[format=string]{AOO}, \answer[format=string]{AOA}, \answer[format=string]{AOB}, \answer[format=string]{AOC}$
%\item What number comes immediately before $AO$?  $\answer[format=string]{F}$
%\item Before $ABO$?  $\answer[format=string]{AAF}$
%\item Before $EOFFA$? $\answer[format=string]{EOFFO}$
%\end{enumerate}
%\end{problem} 

\begin{problem}Now, suppose that you meet a hermit who tallies by
  thirteens. Demonstrate the hermit's counting below.   (Note: For bases greater than ten, the convention is to use A for ten, B for eleven, and so on.)
  
$8, 9, \answer[format=string]{A}, \answer[format=string]{B}, \answer[format=string]{C}, \answer[format=string]{10}, \answer[format=string]{11}, \answer[format=string]{12}, \dots, \\
18, \answer[format=string]{19}, \answer[format=string]{1A}, \answer[format=string]{1B}, \answer[format=string]{1C}, \answer[format=string]{20}, \answer[format=string]{21}$

\end{problem} 


\end{document}