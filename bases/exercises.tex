
\documentclass[nooutcomes]{ximera}
%\documentclass[space,handout,nooutcomes]{ximera}

% For preamble materials

\graphicspath{
  {./}
  {algorithms/}
  {../algorithms/}
}


%%% This set of code is all of our user defined commands
\newcommand{\bysame}{\mbox{\rule{3em}{.4pt}}\,}
\newcommand{\N}{\mathbb N}
\newcommand{\C}{\mathbb C}
\newcommand{\W}{\mathbb W}
\newcommand{\Z}{\mathbb Z}
\newcommand{\Q}{\mathbb Q}
\newcommand{\R}{\mathbb R}
\newcommand{\A}{\mathbb A}
\newcommand{\D}{\mathcal D}
\newcommand{\F}{\mathcal F}
\newcommand{\ph}{\varphi}
\newcommand{\ep}{\varepsilon}
\newcommand{\aph}{\alpha}
\newcommand{\QM}{\begin{center}{\huge\textbf{?}}\end{center}}

\renewcommand{\le}{\leqslant}
\renewcommand{\ge}{\geqslant}
\renewcommand{\a}{\wedge}
\renewcommand{\v}{\vee}
\renewcommand{\l}{\ell}
\newcommand{\mat}{\mathsf}
\renewcommand{\vec}{\mathbf}
\renewcommand{\subset}{\subseteq}
\renewcommand{\supset}{\supseteq}
\renewcommand{\emptyset}{\varnothing}
\newcommand{\xto}{\xrightarrow}
\renewcommand{\qedsymbol}{$\blacksquare$}
\newcommand{\bibname}{References and Further Reading}
\renewcommand{\bar}{\protect\overline}
\renewcommand{\hat}{\protect\widehat}
\renewcommand{\tilde}{\widetilde}
\newcommand{\tri}{\triangle}
\newcommand{\minipad}{\vspace{1ex}}
\newcommand{\leftexp}[2]{{\vphantom{#2}}^{#1}{#2}}

%% More user defined commands
\renewcommand{\epsilon}{\varepsilon}
\renewcommand{\theta}{\vartheta} %% only for kmath
\renewcommand{\l}{\ell}
\renewcommand{\d}{\, d}
\newcommand{\ddx}{\frac{d}{dx}}
\newcommand{\dydx}{\frac{dy}{dx}}


\usepackage{bigstrut}


\newenvironment{sectionOutcomes}{}{}

\usepackage{array}
%\setlength{\extrarowheight}{-.2cm}   % Commented out by Findell to fix table headings.  Was this for typesetting division?  
\newdimen\digitwidth
\settowidth\digitwidth{9}
\def~{\hspace{\digitwidth}}
\def\divrule#1#2{
\noalign{\moveright#1\digitwidth
\vbox{\hrule width#2\digitwidth}}}


\title{Home Base}
\author{Bart Snapp and Brad Findell}
\begin{document}
\begin{abstract}
Problems about numbers in various bases. 
\end{abstract}
\maketitle

\begin{problem}Explain why the following ``joke'' is ``funny:'' \textit{There
  are $10$ types of people in the world. Those who understand base two
  and those who don't.}
\end{problem} 

\begin{problem}You meet some Tripod aliens, they tally by threes. Thankfully
  for everyone involved, they use the symbols $0$, $1$, and $2$. 
\begin{enumerate}
\item Can you explain how a Tripod would count from $11$ to $201$? Be
  sure to carefully explain what number comes after $22$.
\item What number comes immediately before $10$?  $210$? $20110$?
  Explain your reasoning.
\end{enumerate}
\end{problem} 

\begin{problem}You meet some people who tally by sevens. They use the symbols
  $O$, $A$, $B$, $C$, $D$, $E$, and $F$. 
\begin{enumerate}
\item What do the individual symbols $O$, $A$, $B$, $C$, $D$, $E$, and
  $F$ mean?
\item Can you explain how they would count from $DD$ to $AOC$? Be sure
  to carefully explain what number comes after $FF$.
\item What number comes immediately before $AO$?  $ABO$? $EOFFA$?
  Explain your reasoning.
\end{enumerate}
\end{problem} 

\begin{problem}Now, suppose that you meet a hermit who tallies by
  thirteens. Explain how he might count. Give some relevant and
  revealing examples.
% The following problem used to begin $1/6$ of $30$ is $4$, but that allowed  
% proportional reasoning to give the correct answer.  The numbers have been changed.  
\end{problem} 

\begin{problem}While visiting Mos Eisley spaceport, you stop by Chalmun's
  Cantina. After you sit down, you notice that one of the other aliens
  is holding a discussion on fractions. Much to your surprise, they
  explain that $1/6$ of $36$ is $7$. You are unhappy with this,
  knowing that $1/6$ of $36$ is in fact $6$, yet their audience seems
  to agree with it, not you. Next the alien challenges its audience by
  asking, ``What is $1/4$ of $10$?'' What is the correct answer to
  this question, and how many fingers do the aliens have? Explain your
  reasoning.
\end{problem} 

\begin{problem}When the first Venusian to visit Earth attended a sixth grade
  class, it watched the teacher show that
\[
\frac{3}{12} = \frac{1}{4}.
\]
``How strange,'' thought the Venusian. ``On Venus, $\frac{4}{12} =
\frac{1}{4}$.'' What base do Venusians use? Explain your reasoning.
\end{problem} \begin{problem}When the first Martian to visit Earth attended a high school
  algebra class, it watched the teacher show that the only solution of
  the equation
\[
5x^2-50x+125 = 0
\]
is $x = 5$.

``How strange,'' thought the Martian. ``On Mars, $x = 5$ is a solution
of this equation, but there also is another solution.'' If Martians
have more fingers than humans, how many fingers do Martians have on both hands?
Explain your reasoning.

%\begin{teachingnote}
%Here you cannot factor---you must first convert to base $b$.
%\end{teachingnote}

\end{problem} 

\begin{problem}In one of your many space-time adventures, you see the equation
\[
\frac{3}{10} + \frac{4}{13} = \frac{21}{20}
\]
written on a napkin. How many fingers did the beast who wrote this
have? Explain your reasoning.
\end{problem} 

\begin{problem}What is the smallest number of weights needed to produce every
  integer-valued mass from $0$ grams to say $n$ grams? Explain your
  reasoning.
\end{problem} 

\begin{problem}Starting at zero, how high can you count using just your
  fingers?
\begin{enumerate}
\item Explain how to count to $10$.
\item Explain how to count to $35$.
\item Explain how to count to $1023$.
\item Explain how to count to $59048$.
\item Can you count even higher?
\end{enumerate}
Explain your reasoning.

\end{problem}

\end{document}