
\documentclass[nooutcomes]{ximera}
%\documentclass[space,handout,nooutcomes]{ximera}

% For preamble materials

\usepackage{pgf,tikz}
\usepackage{mathrsfs}
\usetikzlibrary{arrows}
\usepackage{framed}
\usepackage{amsmath}
%\pgfplotsset{compat=1.16}

\graphicspath{
  {./}
  {algorithms/}
  {../algorithms/}
}

\pdfOnly{\renewenvironment{image}[1][]{\begin{center}}{\end{center}}}

%%% This set of code is all of our user defined commands
\newcommand{\bysame}{\mbox{\rule{3em}{.4pt}}\,}
\newcommand{\N}{\mathbb N}
\newcommand{\C}{\mathbb C}
\newcommand{\W}{\mathbb W}
\newcommand{\Z}{\mathbb Z}
\newcommand{\Q}{\mathbb Q}
\newcommand{\R}{\mathbb R}
\newcommand{\A}{\mathbb A}
\newcommand{\D}{\mathcal D}
\newcommand{\F}{\mathcal F}
\newcommand{\ph}{\varphi}
\newcommand{\ep}{\varepsilon}
\newcommand{\aph}{\alpha}
\newcommand{\QM}{\begin{center}{\huge\textbf{?}}\end{center}}

\renewcommand{\le}{\leqslant}
\renewcommand{\ge}{\geqslant}
\renewcommand{\a}{\wedge}
\renewcommand{\v}{\vee}
\renewcommand{\l}{\ell}
\newcommand{\mat}{\mathsf}
\renewcommand{\vec}{\mathbf}
\renewcommand{\subset}{\subseteq}
\renewcommand{\supset}{\supseteq}
\renewcommand{\emptyset}{\varnothing}
\newcommand{\xto}{\xrightarrow}
\renewcommand{\qedsymbol}{$\blacksquare$}
\newcommand{\bibname}{References and Further Reading}
\renewcommand{\bar}{\protect\overline}
\renewcommand{\hat}{\protect\widehat}
\renewcommand{\tilde}{\widetilde}
\newcommand{\tri}{\triangle}
\newcommand{\minipad}{\vspace{1ex}}
\newcommand{\leftexp}[2]{{\vphantom{#2}}^{#1}{#2}}

%% More user defined commands
\renewcommand{\epsilon}{\varepsilon}
\renewcommand{\theta}{\vartheta} %% only for kmath
\renewcommand{\l}{\ell}
\renewcommand{\d}{\, d}
\newcommand{\ddx}{\frac{d}{dx}}
\newcommand{\dydx}{\frac{dy}{dx}}


\usepackage{bigstrut}


\newenvironment{sectionOutcomes}{}{}

\usepackage{array}
%\setlength{\extrarowheight}{-.2cm}   % Commented out by Findell to fix table headings.  Was this for typesetting division?  
\newdimen\digitwidth
\settowidth\digitwidth{9}
\def~{\hspace{\digitwidth}}
\def\divrule#1#2{
\noalign{\moveright#1\digitwidth
\vbox{\hrule width#2\digitwidth}}}


\title{Solving Equations}
\author{Bart Snapp and Brad Findell and Jenny Sheldon}
\begin{document}
\begin{abstract}
Problems about solving equations.
\end{abstract}
\maketitle


%\begin{problem}
%Problem
%\begin{freeResponse}
%\begin{hint}
%Hint
%\end{hint}
%\end{freeResponse}
%\end{problem} 


% Solve equation in one variable of the form $f(x)=g(x)$ by graphing both and looking for intersections.
% Solve literal equations (linear, quadratic, power)

\begin{problem}
Jess is solving the equation $6x+13 = 25$.  Here is their work.

\[
25 - 13 = 12 \div 6 = 2
\]
\end{problem}

What is the issue with this work?
\begin{multipleChoice}
	\choice{The algebra is incorrect.}
	\choice[correct]{The equals sign does not mean equal here.}
	\choice{The solution is not related to the original equation.}
	\choice{There is no issue with this work.}
\end{multipleChoice}



\begin{problem}
Complete the following sentences: 
\begin{enumerate}
\item Expressions have \wordChoice{\choice{solutions} \choice[correct]{values} \choice{zeros} \choice{roots}}. 
\item Equations have \wordChoice{\choice[correct]{solutions} \choice{values} \choice{zeros} \choice{roots}}.
\item Polynomials have \wordChoice{\choice{solutions} \choice{values} \choice{zeros} \choice[correct]{roots}}.
\item Functions have \wordChoice{\choice{solutions} \choice[correct]{zeros} \choice{roots}}.
\end{enumerate}

\begin{problem}
Correct!  

Solutions of equations are the values of the variables that make the equation $\answer[format=string]{true}$.  

And zeros of functions are input values that give output values of $\answer{0}$.  

And these ideas are connected:  The $\answer[format=string]{zeros}$ of a function $f$ are 
the $\answer[format=string]{solutions}$ to the equation $f(x)=0$.  
And the $\answer[format=string]{roots}$ of a polynomial $p(x)$ are $\answer[format=string]{zeros}$ of the polynomial function $p$.   
\end{problem}
\end{problem}



\begin{problem}
Give a polynomial $p(x)$ whose leading coefficient is $1$, and which has $x=12$ and $x=-1$ as roots (and no other roots).

\begin{prompt}
	$p(x) = \answer[given]{(x-12)(x+1)}$
\end{prompt}
\end{problem}


\begin{problem}
Give a polynomial $p(x)$ whose leading coefficient is $3$, and which has $x=\frac{2}{3}$ and $x=2$ as roots (and no other roots).

\begin{prompt}
	$p(x) = \answer[given]{(3x-2)(x-2)}$
\end{prompt}
\end{problem}



\begin{problem}
Give a polynomial $p(x)$ of degree 3 whose leading coefficient is $1$, and which has $x=-5$ as a root (and no other roots).

\begin{prompt}
	$p(x) = \answer[given]{(x+5)^3}$
\end{prompt}
\end{problem}




\begin{problem}
Give a polynomial $p(x)$ of degree 4 whose leading coefficient is $1$, and which has $x=8$, $x=1+\sqrt{3}$ and $x = 1-\sqrt{3}$ as roots (and no other roots).

\begin{prompt}
	$p(x) = \answer[given]{(x-8)^2(x-(1+\sqrt{3}))(x-(1-\sqrt{3}))}$
\end{prompt}
\end{problem}



\begin{problem}
Solve the problem below by completing the square.  Practice drawing a diagram to help! Enter your answers from least to greatest.
\[
x^2 + 8x = 20
\]
To complete the square, add $\answer{16}$ to both sides.  
\begin{problem}
Then take the $\answer[format=string]{square root}$ (two words) of both sides, to yield
\begin{prompt}
$\answer{x+4} = \pm \answer{\sqrt{20+16}}$.
\end{prompt}

So the solutions are, from least to greatest,
\begin{prompt}
	$\answer[given]{-10}$, $\answer[given]{2}$.
\end{prompt}
\end{problem}
\end{problem}

\begin{problem}
Solve the problem below by completing the square.  Practice drawing a diagram to help! Enter your answers from least to greatest.
\[
x^2 + 3x = 5
\]
To complete the square, add $\answer{\frac{9}{4}}$ to both sides.  
\begin{problem}
Then take the $\answer[format=string]{square root}$ (two words) of both sides, to yield
\begin{prompt}
$\answer{x+\frac{3}{2}} = \pm \answer{\sqrt{5+\frac{9}{4}}}$.
\end{prompt}

So the solutions are, from least to greatest,  
\begin{prompt}
	$\answer[given]{-\frac{3}{2}-\frac{\sqrt{29}}{2}}$, $\answer[given]{-\frac{3}{2}+\frac{\sqrt{29}}{2}}$. 
\end{prompt}
\end{problem}
\end{problem}


%\begin{problem}
%Solve the problem below by completing the square.  Practice drawing a diagram to help! Enter your answers from least to greatest.
%\[
%4x^2 + 9x = 3
%\]
%\begin{prompt}
%	$\answer[given]{-\frac{9}{8} - \frac{\sqrt{129}}{8}}$, $\answer[given]{\frac{\sqrt{129}}{8} - \frac{9}{8}}$
%\end{prompt}
%\end{problem}



\begin{problem}
According to the Fundamental Theorem of Algebra, how many roots should the polynomial $p(x) = x^4 - 3x^3 + x - 2$ have?
\begin{prompt}
	$\answer[given]{4}$
\end{prompt}
\begin{hint}
	Remember that the Fundamental Theorem of Algebra counts complex roots (real and imaginary), and also repeated roots.
\end{hint}
\end{problem}



\begin{problem}
According to the Fundamental Theorem of Algebra, how many roots should the polynomial $p(x) = x^{16} - 1$ have?
\begin{prompt}
	$\answer[given]{16}$
\end{prompt}
\end{problem}

\begin{problem}
State the Polynomial Division Theorem: 

Given a polynomial $p(x)$ and a divisor $d(x)\ne 0$, there exist polynomials $q(x)$ and $r(x)$ such that 
\[
p(x) = \answer{q(x)}\cdot d(x)+\answer{r(x)}
\]
with the degree of $r(x)$ 
\wordChoice{\choice{$=$} \choice[correct]{$<$} \choice{$\le$} \choice{$\ne$} \choice{$>$}}
the degree of $d(x)$. 
\end{problem}

\begin{problem}
By the division theorem for polynomials, given $p(x)$ and a linear polynomial $(x-a)$, there exists a quotient polynomial $q(x)$ and a remainder $r$ so that $p(x)= q(x)\cdot\left(\answer{x-a}\right) + \answer{r}$.  

From this, the Remainder Theorem states that $p(a)=\answer{r}$.  And the Factor Theorem states that $(x-a)$ is a factor of $p(x)$ exactly when $p(a)=\answer{0}$.
\end{problem}

\begin{problem}
The Rational Root Theorem says that if $\pm \frac{a}{b}$ (written in lowest terms) is a root of a polynomial with 
\wordChoice{\choice{counting number} \choice[correct]{integer} \choice{rational} \choice{real}} 
coefficients, then $a$ must be a factor of the 
\wordChoice{\choice{leading} \choice{quadratic} \choice{linear} \choice[correct]{constant}} term, and $b$ must be a factor of the 
\wordChoice{\choice[correct]{leading} \choice{quadratic} \choice{linear} \choice{constant}} 
term.
\end{problem}

\begin{problem}
By the Rational Root Theorem, which of the following \textbf{could} be rational roots of $p(x) = x^3 + 2x^2 - 8x + 4$?   
(Do not solve this problem by plugging the values into the polynomial!)
\begin{selectAll}
	\choice[correct]{$1$}
	\choice[correct]{$-1$}
	\choice{$\frac{3}{2}$}
	\choice{$-\frac{2}{3}$}
	\choice[correct]{$-4$}
	\choice{$\frac{1}{4}$}
\end{selectAll}
\end{problem}




\begin{problem}
By the Rational Root Theorem, which of the following \textbf{could} be rational roots of $p(x) = 12x^8 + 5x^7 + 3x^5 + 14x^3 - x + 20$?
(Do not solve this problem by plugging the values into the polynomial!)
\begin{selectAll}
	\choice{$3$}
	\choice[correct]{$-1$}
	\choice{$\frac{5}{7}$}
	\choice[correct]{$-\frac{5}{3}$}
	\choice[correct]{$-4$}
	\choice[correct]{$\frac{1}{4}$}
\end{selectAll}
\end{problem}


%\begin{problem}
%Find all solutions to the equation $x^5-31x^4+310x^3-1240x^2+1984x-1024=0$. 
%\begin{hint} 
%The Rational Root Theorem combined with some division of polynomials might help!
%\end{hint}
%Enter your answers in order from least to greatest.
%\begin{prompt}
%$\answer[given]{1}$, $\answer[given]{2}$, $\answer[given]{4}$, $\answer[given]{8}$, $\answer[given]{16}$
%\end{prompt}
%\end{problem}
%
%
%\begin{problem}
%Find all solutions to the equation $x^5-28x^4+288x^3-1358x^2+2927x-2310=0$. 
%\begin{hint} 
%The Rational Root Theorem combined with some division of polynomials might help!
%\end{hint}
%Enter your answers in order from least to greatest.
%\begin{prompt}
%$\answer[given]{2}$, $\answer[given]{3}$, $\answer[given]{5}$, $\answer[given]{7}$, $\answer[given]{11}$
%\end{prompt}
%\end{problem}


%\begin{problem}
%Find all solutions to the equation $x^5-x^4-25x^3+x^2+168x+144=0$. 
%\begin{hint} 
%The Rational Root Theorem combined with some division of polynomials might help!
%\end{hint}
%\begin{hint}
%The same root can appear more than once.  
%\end{hint}
%Enter your answers in order from least to greatest.
%\begin{prompt}
%$\answer[given]{-3}$, $\answer[given]{-3}$, $\answer[given]{-1}$, $\answer[given]{4}$, $\answer[given]{4}$
%\end{prompt}
%\end{problem}

%\begin{problem}
%Find all solutions to the equation $x^4+3x^3-13x^2-51x-36=0$. 
%\begin{hint} 
%The Rational Root Theorem combined with some division of polynomials might help!
%\end{hint}
%\begin{hint}
%The same root can appear more than once.  
%\end{hint}
%Enter your answers in order from least to greatest.
%\begin{prompt}
%$\answer[given]{-3}$, $\answer[given]{-3}$, $\answer[given]{-1}$, and $\answer[given]{4}$. 
%\end{prompt}
%\end{problem}

\begin{problem}
Find all solutions to the equation $2x^3-9x^2+8x+3=0$. 
\begin{hint} 
The Rational Root Theorem combined with some division of polynomials might help!
\end{hint}
Enter your answers in order from least to greatest.
\begin{prompt}
$\answer[given]{(3-\sqrt{17})/4}$, $\answer[given]{(3-\sqrt{17})/4}$, $\answer[given]{3}$.
\end{prompt}
\end{problem}


\begin{problem}
Find all solutions to the equation $2x^3-6x^2+x+6=0$. 
\begin{hint} 
The Rational Root Theorem combined with some division of polynomials might help!
\end{hint}
Enter your answers in order from least to greatest.
\begin{prompt}
$\answer[given]{(1-\sqrt{7})/2}$, $\answer[given]{(1+\sqrt{7})/2}$, $\answer[given]{2}$.
\end{prompt}
\end{problem}

\begin{problem}

Solve the equation $a^2+3ba = 4b - c$ for $b$. 

How many solutions? $\answer{1}$.

\begin{problem}
Correct!  The equation is a $\answer[format=string]{linear}$ equation in $b$.  

The solution is $b=\answer{\frac{a^2 + c}{4 - 3 a}}$, as long as $a\ne\answer{\frac{4}{3}}$.  
\end{problem}
\end{problem}


\begin{problem}
Solve the equation $a^2+3ba = 4b - c$ for $a$. 

How many solutions? $\answer{2}$.
\begin{problem}
Correct!  There are two solutions because the equation is $\answer[format=string]{quadratic}$ in $a$.  

Here the solutions are $a=\answer{-\frac{3b}{2}}\pm \answer{\frac{\sqrt{(3b)^2 + 4(4b-c)}}{2}}$.  
\end{problem}
\end{problem}



\end{document}



