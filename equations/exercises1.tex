
\documentclass[nooutcomes]{ximera}
%\documentclass[space,handout,nooutcomes]{ximera}

% For preamble materials

\graphicspath{
  {./}
  {algorithms/}
  {../algorithms/}
}


%%% This set of code is all of our user defined commands
\newcommand{\bysame}{\mbox{\rule{3em}{.4pt}}\,}
\newcommand{\N}{\mathbb N}
\newcommand{\C}{\mathbb C}
\newcommand{\W}{\mathbb W}
\newcommand{\Z}{\mathbb Z}
\newcommand{\Q}{\mathbb Q}
\newcommand{\R}{\mathbb R}
\newcommand{\A}{\mathbb A}
\newcommand{\D}{\mathcal D}
\newcommand{\F}{\mathcal F}
\newcommand{\ph}{\varphi}
\newcommand{\ep}{\varepsilon}
\newcommand{\aph}{\alpha}
\newcommand{\QM}{\begin{center}{\huge\textbf{?}}\end{center}}

\renewcommand{\le}{\leqslant}
\renewcommand{\ge}{\geqslant}
\renewcommand{\a}{\wedge}
\renewcommand{\v}{\vee}
\renewcommand{\l}{\ell}
\newcommand{\mat}{\mathsf}
\renewcommand{\vec}{\mathbf}
\renewcommand{\subset}{\subseteq}
\renewcommand{\supset}{\supseteq}
\renewcommand{\emptyset}{\varnothing}
\newcommand{\xto}{\xrightarrow}
\renewcommand{\qedsymbol}{$\blacksquare$}
\newcommand{\bibname}{References and Further Reading}
\renewcommand{\bar}{\protect\overline}
\renewcommand{\hat}{\protect\widehat}
\renewcommand{\tilde}{\widetilde}
\newcommand{\tri}{\triangle}
\newcommand{\minipad}{\vspace{1ex}}
\newcommand{\leftexp}[2]{{\vphantom{#2}}^{#1}{#2}}

%% More user defined commands
\renewcommand{\epsilon}{\varepsilon}
\renewcommand{\theta}{\vartheta} %% only for kmath
\renewcommand{\l}{\ell}
\renewcommand{\d}{\, d}
\newcommand{\ddx}{\frac{d}{dx}}
\newcommand{\dydx}{\frac{dy}{dx}}


\usepackage{bigstrut}


\newenvironment{sectionOutcomes}{}{}

\usepackage{array}
%\setlength{\extrarowheight}{-.2cm}   % Commented out by Findell to fix table headings.  Was this for typesetting division?  
\newdimen\digitwidth
\settowidth\digitwidth{9}
\def~{\hspace{\digitwidth}}
\def\divrule#1#2{
\noalign{\moveright#1\digitwidth
\vbox{\hrule width#2\digitwidth}}}


\title{Solving Equations}
\author{Bart Snapp and Brad Findell and Jenny Sheldon}
\begin{document}
\begin{abstract}
Problems about solving equations.
\end{abstract}
\maketitle


%\begin{problem}
%Problem
%\begin{freeResponse}
%\begin{hint}
%Hint
%\end{hint}
%\end{freeResponse}
%\end{problem} 


\begin{problem}
Jess is solving the equation $6x+13 = 25$.  Here is their work.

\[
25 - 13 = 12 \div 6 = 2
\]
\end{problem}

What is the issue with this work?
\begin{multipleChoice}
	\choice{The algebra is incorrect.}
	\choice[correct]{The equals sign does not mean equal here.}
	\choice{The solution is not related to the original equation.}
	\choice{There is no issue with this work.}
\end{multipleChoice}



\begin{problem}
Give a polynomial $p(x)$ whose leading coefficient is $1$, and which has $x=12$ and $x=-1$ as roots (and no other roots).

\begin{prompt}
	$p(x) = \answer[given]{(x-12)(x+1)}$
\end{prompt}
\end{problem}



\begin{problem}
Give a polynomial $p(x)$ whose leading coefficient is $3$, and which has $x=\frac{2}{3}$ and $x=2$ as roots (and no other roots).

\begin{prompt}
	$p(x) = \answer[given]{(3x-2)(x-2)}$
\end{prompt}
\end{problem}



\begin{problem}
Give a polynomial $p(x)$ of degree 3 whose leading coefficient is $1$, and which has $x=-5$ as a root (and no other roots).

\begin{prompt}
	$p(x) = \answer[given]{(x+5)^3}$
\end{prompt}
\end{problem}




\begin{problem}
Give a polynomial $p(x)$ of degree 4 whose leading coefficient is $1$, and which has $x=8$, $x=1+\sqrt{3}$ and $x = 1-\sqrt{3}$ as roots (and no other roots).

\begin{prompt}
	$p(x) = \answer[given]{(x-8)^2(x-(1+\sqrt{3}))(x-(1-\sqrt{3}))}$
\end{prompt}
\end{problem}



\begin{problem}
Solve the problem below by completing the square.  Practice drawing a diagram to help! Enter your answers from smallest to largest.
\[
x^2 + 8x = 20
\]
\begin{prompt}
	$\answer[given]{-10}$, $\answer[given]{2}$
\end{prompt}
\end{problem}



\begin{problem}
Solve the problem below by completing the square.  Practice drawing a diagram to help! Enter your answers from smallest to largest.
\[
4x^2 + 9x = 3
\]
\begin{prompt}
	$\answer[given]{-\frac{9}{8} - \frac{\sqrt{129}}{8}}$, $\answer[given]{\frac{\sqrt{129}}{8} - \frac{9}{8}}$
\end{prompt}
\end{problem}



\begin{problem}
According to the Fundamental Theorem of Algebra, how many roots should the polynomial $p(x) = x^4 - 3x^3 + x - 2$ have?
\begin{prompt}
	$\answer[given]{4}$
\end{prompt}
\begin{hint}
	Remember that the Fundamental Theorem of Algebra counts real and complex roots, and also repeated roots.
\end{hint}
\end{problem}



\begin{problem}
According to the Fundamental Theorem of Algebra, how many roots should the polynomial $p(x) = x^{16} - 1$ have?
\begin{prompt}
	$\answer[given]{16}$
\end{prompt}
\end{problem}



\begin{problem}
The Rational Root Theorem says that if $\pm \frac{a}{b}$ is a root of a polynomial (written in lowest terms), then $a$ must be a factor of the constant term, and $b$ must be a factor of the leading term.

For the polynomial $p(x) = x^3 + 2x^2 - 8x + 4$, which of the following could be rational roots of $p(x)$?   (Do not solve this problem by plugging in the answers to the polynomial!)
\begin{selectAll}
	\choice[correct]{$1$}
	\choice[correct]{$-1$}
	\choice{$\frac{3}{2}$}
	\choice{$-\frac{2}{3}$}
	\choice[correct]{$-4$}
	\choice{$\frac{1}{4}$}
\end{selectAll}
\end{problem}




\begin{problem}
The Rational Root Theorem says that if $\pm \frac{a}{b}$ is a root of a polynomial (written in lowest terms), then $a$ must be a factor of the constant term, and $b$ must be a factor of the leading term.

For the polynomial $p(x) = 12x^8 + 5x^7 + 3x^5 + 14x^3 - x + 20$, which of the following could be rational roots of $p(x)$?   (Do not solve this problem by plugging in the answers to the polynomial!)
\begin{selectAll}
	\choice{$3$}
	\choice[correct]{$-1$}
	\choice{$\frac{5}{7}$}
	\choice[correct]{$-\frac{5}{3}$}
	\choice[correct]{$-4$}
	\choice[correct]{$\frac{1}{4}$}
\end{selectAll}
\end{problem}


\end{document}



