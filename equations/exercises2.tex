
\documentclass[nooutcomes]{ximera}
%\documentclass[space,handout,nooutcomes]{ximera}

% For preamble materials

\usepackage{pgf,tikz}
\usepackage{mathrsfs}
\usetikzlibrary{arrows}
\usepackage{framed}
\usepackage{amsmath}
%\pgfplotsset{compat=1.16}

\graphicspath{
  {./}
  {algorithms/}
  {../algorithms/}
}

\pdfOnly{\renewenvironment{image}[1][]{\begin{center}}{\end{center}}}

%%% This set of code is all of our user defined commands
\newcommand{\bysame}{\mbox{\rule{3em}{.4pt}}\,}
\newcommand{\N}{\mathbb N}
\newcommand{\C}{\mathbb C}
\newcommand{\W}{\mathbb W}
\newcommand{\Z}{\mathbb Z}
\newcommand{\Q}{\mathbb Q}
\newcommand{\R}{\mathbb R}
\newcommand{\A}{\mathbb A}
\newcommand{\D}{\mathcal D}
\newcommand{\F}{\mathcal F}
\newcommand{\ph}{\varphi}
\newcommand{\ep}{\varepsilon}
\newcommand{\aph}{\alpha}
\newcommand{\QM}{\begin{center}{\huge\textbf{?}}\end{center}}

\renewcommand{\le}{\leqslant}
\renewcommand{\ge}{\geqslant}
\renewcommand{\a}{\wedge}
\renewcommand{\v}{\vee}
\renewcommand{\l}{\ell}
\newcommand{\mat}{\mathsf}
\renewcommand{\vec}{\mathbf}
\renewcommand{\subset}{\subseteq}
\renewcommand{\supset}{\supseteq}
\renewcommand{\emptyset}{\varnothing}
\newcommand{\xto}{\xrightarrow}
\renewcommand{\qedsymbol}{$\blacksquare$}
\newcommand{\bibname}{References and Further Reading}
\renewcommand{\bar}{\protect\overline}
\renewcommand{\hat}{\protect\widehat}
\renewcommand{\tilde}{\widetilde}
\newcommand{\tri}{\triangle}
\newcommand{\minipad}{\vspace{1ex}}
\newcommand{\leftexp}[2]{{\vphantom{#2}}^{#1}{#2}}

%% More user defined commands
\renewcommand{\epsilon}{\varepsilon}
\renewcommand{\theta}{\vartheta} %% only for kmath
\renewcommand{\l}{\ell}
\renewcommand{\d}{\, d}
\newcommand{\ddx}{\frac{d}{dx}}
\newcommand{\dydx}{\frac{dy}{dx}}


\usepackage{bigstrut}


\newenvironment{sectionOutcomes}{}{}

\usepackage{array}
%\setlength{\extrarowheight}{-.2cm}   % Commented out by Findell to fix table headings.  Was this for typesetting division?  
\newdimen\digitwidth
\settowidth\digitwidth{9}
\def~{\hspace{\digitwidth}}
\def\divrule#1#2{
\noalign{\moveright#1\digitwidth
\vbox{\hrule width#2\digitwidth}}}


\title{Complex Numbers}
\author{Bart Snapp and Brad Findell and Jenny Sheldon}
\begin{document}
\begin{abstract}
Problems about complex numbers.
\end{abstract}
\maketitle


%\begin{problem}
%Problem
%\begin{freeResponse}
%\begin{hint}
%Hint
%\end{hint}
%\end{freeResponse}
%\end{problem} 

\begin{problem}
A \textbf{rational number} is any number that can be expressed as $\answer{\frac{a}{b}}$ where $a$ and $b$ are $\answer[format=string]{integers}$ and $b$ \wordChoice{\choice{$=$} \choice{$<$} \choice{$>$} \choice[correct]{$\ne$}} $\answer{0}$.
\end{problem}

\begin{problem}
The $\answer[format=string]{real}$ numbers are all of the numbers on the number line.
\begin{problem}
Correct!  This is why the number line is sometimes called the \textbf{real line}.  

Real numbers that are not rational are said to be $\answer[format=string]{irrational}$. 
\end{problem}
\end{problem}

\begin{problem}
Complex numbers can be expressed as $a+bi$, where $a$ and $b$ are $\answer[format=string]{real}$ numbers.  
\end{problem}

\begin{problem}
Which of the following are rational numbers?  Select all that apply.

\begin{selectAll}
	\choice[correct]{$7$}
	\choice{$e$}
	\choice{$\frac{\pi^2}{6}$}
	\choice[correct]{$\frac{18}{11}$}
	\choice{$8-3i$}
	\choice{$\sqrt{-17}$}
	\choice{$\sqrt[3]{-2}$}
\end{selectAll}
\end{problem}



\begin{problem}
Which of the following are real numbers?  Select all that apply.

\begin{selectAll}
	\choice[correct]{$7$}
	\choice[correct]{$e$}
	\choice[correct]{$\frac{\pi^2}{6}$}
	\choice[correct]{$\frac{18}{11}$}
	\choice{$8-3i$}
	\choice{$\sqrt{-17}$}
	\choice[correct]{$\sqrt[3]{-2}$}
\end{selectAll}
\end{problem}



\begin{problem}
Which of the following are complex numbers?  Select all that apply.

\begin{selectAll}
	\choice[correct]{$7$}
	\choice[correct]{$e$}
	\choice[correct]{$\frac{\pi^2}{6}$}
	\choice[correct]{$\frac{18}{11}$}
	\choice[correct]{$8-3i$}
	\choice[correct]{$\sqrt{-17}$}
	\choice[correct]{$\sqrt[3]{-2}$}
\end{selectAll}
\begin{hint}
Complex numbers can be expressed as $a+bi$, where $a$ and $b$ are real numbers.  
\end{hint}
\begin{problem}
Correct! The complex numbers \textbf{include} the real numbers.

A complex number that is \textbf{not real} is typically called 
$\answer[format=string]{imaginary}$. 

Suppose $z=a+bi$, where $a$ and $b$ are real numbers.  The number $z$ is called \textbf{real} when $\answer{b}=0$ or 
\textbf{pure imaginary} when $\answer{a}=0$.  
\end{problem}
\end{problem}


\begin{problem}
Assuming none of the numbers involved are zero, select all operations below that must produce an irrational number.

\begin{selectAll}
	\choice{rational $+$ rational}
	\choice[correct]{rational $+$ irrational}
	\choice{irrational $+$ irrational}
	\choice{rational $\times$ rational}
	\choice[correct]{irrational $\times$ rational}
	\choice{irrational $\times$ irrational}
\end{selectAll}
\end{problem}

In the following problems express the result in the form $a+bi$, with $a$ and $b$ real numbers.  

\begin{problem}
Compute $(2+i) + 4 = $ 
\begin{prompt}
	$\answer[given]{6} + \answer[given]{1}\,i$
\end{prompt}
\end{problem}


\begin{problem}
Compute $(-3+4i) - (-8 - i) =$
\begin{prompt}
	$\answer[given]{5} + \answer[given]{5}\,i$
\end{prompt}
\end{problem}


\begin{problem}
Compute $(2-6i) - (3+8i) =$ 
\begin{prompt}
	$\answer[given]{-1} + \answer[given]{-14}\,i$
\end{prompt}
\end{problem}

\begin{javascript}
  a = Math.round(Math.random()*10-5)
  b = Math.round(Math.random()*10-5)
  c = Math.round(Math.random()*10-5)
  d = Math.round(Math.random()*10-5)
\end{javascript}

\begin{problem}
Compute $(\js{a}+\js{b}i)+(\js{c}+\js{d}i) = \begin{prompt} \answer{\js{a+b}} + \answer{\js{c+d}}\,i\end{prompt}$
\end{problem}

\begin{problem}
Compute $(\js{a}+\js{b}i)\cdot(\js{c}+\js{d}i) = \begin{prompt} \answer{\js{ac-bd}} + \answer{\js{ad+bc}}\,i\end{prompt}$
\end{problem}


\begin{problem}
Compute $(2+i) \cdot 4 =$
\begin{prompt}
	$\answer[given]{8} + \answer[given]{4}\,i$
\end{prompt}
\end{problem}



\begin{problem}
Compute $(-3 + 4i) \cdot (-8 - i) =$
\begin{prompt}
	$\answer[given]{28} + \answer[given]{-29}\,i$
\end{prompt}
\end{problem}




\begin{problem}
Compute $(2-6i) \cdot (3+8i)=$
\begin{prompt}
	$\answer[given]{54} + \answer[given]{-2}\,i$
\end{prompt}
\end{problem}


\begin{problem}
Compute $(a+bi) \cdot (a+bi)=$
\begin{prompt}
	$\left(\answer[given]{a^2-b^2}\right) + \left(\answer[given]{2ab}\right)i$
\end{prompt}
\end{problem}

\begin{problem}
Compute $(a+bi) \cdot (a-bi)=$
\begin{prompt}
	$\left(\answer[given]{a^2+b^2}\right) + \left(\answer[given]{0}\right)i$
\end{prompt}
\begin{problem}
Usually the product of two imaginary numbers is an imaginary number.  But here we have multiplied the number $a+bi$ by its
\textbf{conjugate}, $a-bi$, and the result is a $\answer[format=string]{real}$ number because the imaginary part of the product is $\answer{0}$.   
\end{problem}
\end{problem}


\begin{problem}
Compute $\frac{1}{2+i} = $
\begin{prompt}
	$\answer[given]{\frac{2}{5}} + \answer[given]{-\frac{1}{5}}\,i$
\end{prompt}
\begin{hint}
Try multiplying the numerator and denominator by something that will make the denominator into an integer.  
\end{hint}
\begin{hint}
Try the complex conjugate of $2+i$.
\end{hint}
\end{problem}



\begin{problem}
Compute $(1-3i) \div (-3+5i)$.
\begin{prompt}
	$\answer[given]{-\frac{18}{34}} + \answer[given]{\frac{4}{34}}\,i$
\end{prompt}
\begin{hint}
Perhaps first compute $\frac{1}{-3+5i}$, and then multiply.
\end{hint}
\end{problem}

\begin{problem}
From your experience with imaginary roots of polynomials, you might guess that if $3+5i$ is a root of the polynomial, then $\answer{3-5i}$ is also a root.    
\begin{problem}
Correct.  More generally, if a polynomial has $\answer[format=string]{real}$ coefficients and $a+bi$ is a root, then $\answer{a-bi}$ is also a root.  In such cases, we say that imaginary roots come as a \textbf{conjugate pair}, $a\pm bi$.  

\end{problem}
\end{problem}


\begin{problem}
Find all solutions to the equation $x^3-3x^2+5x-3=0$. 
\begin{hint} 
The Rational Root Theorem combined with some division of polynomials might help!
\end{hint}
Enter first imaginary solutions (as a conjugate pair), and then real solutions from least to greatest.

\begin{prompt}
$\answer[given]{1} \pm \answer[given]{\sqrt{2}}\,i$, $\answer[given]{1}$
\end{prompt}
\end{problem}


\begin{problem}
Find all solutions to the equation $x^3+2x-3=0$. 
\begin{hint} 
The Rational Root Theorem combined with some division of polynomials might help!
\end{hint}
Enter first imaginary solutions (as a conjugate pair), and then real solutions from least to greatest.

\begin{prompt}
$\answer[given]{-\frac{1}{2}} \pm \answer[given]{\frac{\sqrt{11}}{2}}\,i$, $\answer[given]{1}$
\end{prompt}
\end{problem}


\begin{problem}
Find all solutions to the equation $x^4+6x^3+14x^2+30x+45=0$. 
\begin{hint} 
The Rational Root Theorem combined with some division of polynomials might help!
\end{hint}
Enter first imaginary solutions (as a conjugate pair), and then real solutions from least to greatest.

\begin{prompt}
$\answer[given]{0} \pm \answer[given]{\sqrt{5}}\,i$, $\answer[given]{-3}$
\end{prompt}
\end{problem}




\end{document}



